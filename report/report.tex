% Options for packages loaded elsewhere
\PassOptionsToPackage{unicode}{hyperref}
\PassOptionsToPackage{hyphens}{url}
\PassOptionsToPackage{dvipsnames,svgnames,x11names}{xcolor}
%
\documentclass[
]{scrartcl}

\usepackage{amsmath,amssymb}
\usepackage{iftex}
\ifPDFTeX
  \usepackage[T1]{fontenc}
  \usepackage[utf8]{inputenc}
  \usepackage{textcomp} % provide euro and other symbols
\else % if luatex or xetex
  \usepackage{unicode-math}
  \defaultfontfeatures{Scale=MatchLowercase}
  \defaultfontfeatures[\rmfamily]{Ligatures=TeX,Scale=1}
\fi
\usepackage{lmodern}
\ifPDFTeX\else  
    % xetex/luatex font selection
\fi
% Use upquote if available, for straight quotes in verbatim environments
\IfFileExists{upquote.sty}{\usepackage{upquote}}{}
\IfFileExists{microtype.sty}{% use microtype if available
  \usepackage[]{microtype}
  \UseMicrotypeSet[protrusion]{basicmath} % disable protrusion for tt fonts
}{}
\makeatletter
\@ifundefined{KOMAClassName}{% if non-KOMA class
  \IfFileExists{parskip.sty}{%
    \usepackage{parskip}
  }{% else
    \setlength{\parindent}{0pt}
    \setlength{\parskip}{6pt plus 2pt minus 1pt}}
}{% if KOMA class
  \KOMAoptions{parskip=half}}
\makeatother
\usepackage{xcolor}
\usepackage[left=2.5cm,right=2.5cm,top=3cm,bottom=3cm]{geometry}
\setlength{\emergencystretch}{3em} % prevent overfull lines
\setcounter{secnumdepth}{5}
% Make \paragraph and \subparagraph free-standing
\ifx\paragraph\undefined\else
  \let\oldparagraph\paragraph
  \renewcommand{\paragraph}[1]{\oldparagraph{#1}\mbox{}}
\fi
\ifx\subparagraph\undefined\else
  \let\oldsubparagraph\subparagraph
  \renewcommand{\subparagraph}[1]{\oldsubparagraph{#1}\mbox{}}
\fi


\providecommand{\tightlist}{%
  \setlength{\itemsep}{0pt}\setlength{\parskip}{0pt}}\usepackage{longtable,booktabs,array}
\usepackage{calc} % for calculating minipage widths
% Correct order of tables after \paragraph or \subparagraph
\usepackage{etoolbox}
\makeatletter
\patchcmd\longtable{\par}{\if@noskipsec\mbox{}\fi\par}{}{}
\makeatother
% Allow footnotes in longtable head/foot
\IfFileExists{footnotehyper.sty}{\usepackage{footnotehyper}}{\usepackage{footnote}}
\makesavenoteenv{longtable}
\usepackage{graphicx}
\makeatletter
\def\maxwidth{\ifdim\Gin@nat@width>\linewidth\linewidth\else\Gin@nat@width\fi}
\def\maxheight{\ifdim\Gin@nat@height>\textheight\textheight\else\Gin@nat@height\fi}
\makeatother
% Scale images if necessary, so that they will not overflow the page
% margins by default, and it is still possible to overwrite the defaults
% using explicit options in \includegraphics[width, height, ...]{}
\setkeys{Gin}{width=\maxwidth,height=\maxheight,keepaspectratio}
% Set default figure placement to htbp
\makeatletter
\def\fps@figure{htbp}
\makeatother
% definitions for citeproc citations
\NewDocumentCommand\citeproctext{}{}
\NewDocumentCommand\citeproc{mm}{%
  \begingroup\def\citeproctext{#2}\cite{#1}\endgroup}
\makeatletter
 % allow citations to break across lines
 \let\@cite@ofmt\@firstofone
 % avoid brackets around text for \cite:
 \def\@biblabel#1{}
 \def\@cite#1#2{{#1\if@tempswa , #2\fi}}
\makeatother
\newlength{\cslhangindent}
\setlength{\cslhangindent}{1.5em}
\newlength{\csllabelwidth}
\setlength{\csllabelwidth}{3em}
\newenvironment{CSLReferences}[2] % #1 hanging-indent, #2 entry-spacing
 {\begin{list}{}{%
  \setlength{\itemindent}{0pt}
  \setlength{\leftmargin}{0pt}
  \setlength{\parsep}{0pt}
  % turn on hanging indent if param 1 is 1
  \ifodd #1
   \setlength{\leftmargin}{\cslhangindent}
   \setlength{\itemindent}{-1\cslhangindent}
  \fi
  % set entry spacing
  \setlength{\itemsep}{#2\baselineskip}}}
 {\end{list}}
\usepackage{calc}
\newcommand{\CSLBlock}[1]{\hfill\break\parbox[t]{\linewidth}{\strut\ignorespaces#1\strut}}
\newcommand{\CSLLeftMargin}[1]{\parbox[t]{\csllabelwidth}{\strut#1\strut}}
\newcommand{\CSLRightInline}[1]{\parbox[t]{\linewidth - \csllabelwidth}{\strut#1\strut}}
\newcommand{\CSLIndent}[1]{\hspace{\cslhangindent}#1}

\usepackage[noblocks]{authblk}
\renewcommand*{\Authsep}{, }
\renewcommand*{\Authand}{, }
\renewcommand*{\Authands}{, }
\renewcommand\Affilfont{\small}
\usepackage{lineno}
\usepackage{scrlayer-scrpage}
\rohead{IOTC-2024-WPTT26-XX}
\makeatletter
\@ifpackageloaded{caption}{}{\usepackage{caption}}
\AtBeginDocument{%
\ifdefined\contentsname
  \renewcommand*\contentsname{Table of contents}
\else
  \newcommand\contentsname{Table of contents}
\fi
\ifdefined\listfigurename
  \renewcommand*\listfigurename{List of Figures}
\else
  \newcommand\listfigurename{List of Figures}
\fi
\ifdefined\listtablename
  \renewcommand*\listtablename{List of Tables}
\else
  \newcommand\listtablename{List of Tables}
\fi
\ifdefined\figurename
  \renewcommand*\figurename{Figure}
\else
  \newcommand\figurename{Figure}
\fi
\ifdefined\tablename
  \renewcommand*\tablename{Table}
\else
  \newcommand\tablename{Table}
\fi
}
\@ifpackageloaded{float}{}{\usepackage{float}}
\floatstyle{ruled}
\@ifundefined{c@chapter}{\newfloat{codelisting}{h}{lop}}{\newfloat{codelisting}{h}{lop}[chapter]}
\floatname{codelisting}{Listing}
\newcommand*\listoflistings{\listof{codelisting}{List of Listings}}
\makeatother
\makeatletter
\makeatother
\makeatletter
\@ifpackageloaded{caption}{}{\usepackage{caption}}
\@ifpackageloaded{subcaption}{}{\usepackage{subcaption}}
\makeatother
\ifLuaTeX
  \usepackage{selnolig}  % disable illegal ligatures
\fi
\usepackage{bookmark}

\IfFileExists{xurl.sty}{\usepackage{xurl}}{} % add URL line breaks if available
\urlstyle{same} % disable monospaced font for URLs
\hypersetup{
  pdftitle={Preliminary 2024 stock assessment of yellowfin tuna in the Indian Ocean},
  colorlinks=true,
  linkcolor={blue},
  filecolor={Maroon},
  citecolor={Blue},
  urlcolor={Blue},
  pdfcreator={LaTeX via pandoc}}

\title{Preliminary 2024 stock assessment of yellowfin tuna in the Indian
Ocean}


\author[1,2]{Name 1}
\author[2]{Name 2}

\affil[1]{AZTI, Marine Research, Basque Research and Technology Alliance
(BRTA), Txatxarramendi ugartea z/g, 48395 Sukarrieta (Bizkaia), Spain}
\affil[2]{Affiliation 2}


\date{}
\begin{document}
\maketitle

\renewcommand*\contentsname{Contents}
{
\hypersetup{linkcolor=blue}
\setcounter{tocdepth}{3}
\tableofcontents
}
\newpage{}

\section*{Executive Summary}\label{executive-summary}
\addcontentsline{toc}{section}{Executive Summary}

This report presents a preliminary stock assessment for Indian Ocean
yellowfin tuna (\emph{Thunnus albacares}) using Stock Synthesis 3 (SS3).
The assessment uses an age-structured and spatially-explicit population
model and is fitted to catch rate indices, length-composition data, and
tagging data. The assessment covers 1950 -- 2023 and represents an
update of the previous assessment model, taking into account progress
and improvements made since the previous assessment. The assessment
assumes that the Indian Ocean yellowfin tuna constitute a single
spawning stock, modelled as spatially disaggregated four regions, with
21 fisheries. Standardized CPUE series from the main longline fleets
1975 -- 2020 were included in the models as the relative abundance index
of exploitable biomass in each region. The CPUE indices from EU Purse
seine sets on free schools were included in a subset of models with the
spatial and fleet structure revised to better accommodate the
distribution and size structure of the purse seine fisheries. Indices
based on associative and non-associative dynamics of yellowfin tuna with
floating objects were also available, and the utility of these indices
was examined in the assessment. Tag release and recovery data from the
RTTP-IO program were included in the model to inform abundance,
movement, and mortality rates.

\newpage{}

\section{Introduction}\label{introduction}

Prior to 2008, Indian Ocean (IO) yellowfin tuna (\emph{Thunnus
albacares}) was assessed using methods such as Virtual Population
Analysis (VPA) and production models
(\citeproc{ref-nishidaStockAssessmentYellowfin2007}{Nishida and Shono
2007}, \citeproc{ref-nishidaStockAssessmentYellowfin2005}{2005}). In
2008, a preliminary stock assessment of IO yellowfin tuna was conducted
using MULTIFAN-CL
(\citeproc{ref-langleyPreliminaryStockAssessment2008}{Langley et al.
2008}) enabling the integration of the tag release/recovery data
collected from the large-scale tagging programme conducted in the IO in
the preceding years. The MULTIFAN-CL assessment was revised and updated
in the following years
(\citeproc{ref-langleyStockAssessmentYellowfin2009}{Langley et al.
2009}; \citeproc{ref-langleyStockAssessmentYellowfin2010}{Langley,
Herrera, and Million 2010},
\citeproc{ref-langleyStockAssessmentYellowfin2011}{2011},
\citeproc{ref-langleyStockAssessmentYellowfin2012}{2012}).

In 2015, the assessment of IO yellowfin tuna was implemented
(\citeproc{ref-langleyStockAssessmentYellowfin2015}{Langley 2015}),
which used the Stock Synthesis 3 modelling platform
(\citeproc{ref-methotStockSynthesisBiological2013}{Methot and Wetzel
2013}). SS3 is conceptually very similar to MFCL and the two platforms
have yielded similar results. On basis of that assessment, the yellowfin
tuna stock was determined to be overfished and subject to overfishing.
At its 20th meeting, the Indian Ocean Tuna Commission (IOTC) adopted an
Interim Plan for Rebuilding the Indian Ocean Yellowfin Tuna Stock (Res.
16/01).

The SS3 assessment was updated in 2016
(\citeproc{ref-langleyUpdate2015Indian2016}{Langley 2016}) and was
revised and updated in 2018
(\citeproc{ref-fuPreliminaryIndianOcean2018}{Fu et al. 2018}). These
assessments utilised new composite longline CPUE indices derived from
the main distant water longline fleets, replacing the Japanese longline
CPUE indices used previously. The 2018 assessment also included a
comprehensive analysis of the main assumptions of the stock assessment.
A model ensemble covering major components of structural uncertainty was
used to characterise the stock status. The assessment estimated that the
spawning stock biomass in 2017 was below \(SSB_{MSY}\), and that fishing
mortality was above \(F_{MSY}\). Therefore, the stock status was
determined to remain overfished and experiencing overfishing.

An external review of the 2018 assessment provided recommendations to
improve model parametrisations
(\citeproc{ref-methotRecommendationsConfigurationIndian2019}{Methot
2019}). An attempt was made to update the assessment in 2019, with
extensive investigations of alternative spatial structures, data
weighting and biological parameters
(\citeproc{ref-urtizbereaPreliminaryAssessmentIndian2019}{Urtizberea et
al. 2019}). Further analysis was conducted in 2020 to refine the process
of model selection through an objective scoring system based on
diagnostic metrics
(\citeproc{ref-urtizbereaProvidingScientificAdvice2020}{Urtizberea et
al. 2020}).

The most recent assessment was conducted in 2021
(\citeproc{ref-fuPreliminaryIndianYellowfin2021}{Fu et al. 2021}), which
also used SS3 as the modelling platform and was based on the four area
spatial configuration as in 2018. This recent assessment included an
standardised CPUE series from the main longline fleets as the main
index, but also tested the inclusion of EU purse seine indices,
operating on free schools and floating objects, and an index from the
Maldivian pole and line fishery. A range of exploratory models were
presented to address issues in observational datasets, improve the
stability of the assessment model, and explore the effects of
alternative model assumptions. Overall stock status estimates do not
differ substantially from the 2018 assessment, estimating
\(SSB/SSB_{MSY}=0.78\) and \(F/F_{MSY}=1.27\) for the last model year,
which suggest that the stock is overfished and experiencing overfishing.

An external review of the 2021 assessment was carried out in 2023
(\citeproc{ref-maunderIndependentReviewRecent2023}{Maunder et al.
2023}), and provided a set of recommendations for the next assessment
implementation. This report documents the next iteration of the stock
assessment of the IO yellowfin tuna stock for consideration at the 26th
WPTT meeting. This stock assessment includes fishery and biological data
up to the end of 2023 and its configuration is based on the 2021
assessment. It implements an age- and spatially-structured population
model using SS3 (v3.30.22.1) and incorporates some revisions made by the
last external review.

\section{Background}\label{background}

\subsection{Biology}\label{biology}

Yellowfin tuna is a cosmopolitan species distributed mainly in the
tropical and subtropical oceanic waters of the three major oceans, where
it forms large schools. Spawning occurs mainly from December to March in
lower latitudes with warmer waters and mesoscale oceanographic activity
(\citeproc{ref-muhlingReproductionLarvalBiology2017}{Muhling et al.
2017}), with the main spawning grounds west of 75\(^\circ\)E. However,
spawning activity has also been reported in the Oman Sea
(\citeproc{ref-hosseiniInvestigationsReproductiveBiology2016}{Hosseini
and Kaymaram 2016}), Bay of Bengal
(\citeproc{ref-kumarReproductiveDynamicsYellowfin2022}{Kumar and Ghosh
2022}), off Sri Lanka and the Mozambique Channel, and in the eastern IO
off Australia Nootmorn, Yakoh, and Kawises
(\citeproc{ref-nootmornReproductiveBiologyYellowfin2005}{2005}). The
size at 50\% maturity for this species in the IO was initially estimated
at around 75 cm based on cortical alveolar stage
(\citeproc{ref-zudaireReproductivePotentialYellowfin2013}{Zudaire et al.
2013}), but an updated study suggests that it might be at a larger size
(\(\sim\) 101 cm)
(\citeproc{ref-zudairePreliminaryEstimatesSex2022}{Zudaire et al.
2022}). Tag recoveries provide evidence of large movements of yellowfin
tuna within the western equatorial region; however, few observations of
large-scale transverse movements in the IO have been reported
(\citeproc{ref-gaertnerTagSheddingTropical2015}{Gaertner and Hallier
2015}). Yellowfin dwell preferentially in the surface mixed layer and
the thermocline (\citeproc{ref-pecoraroPuttingAllPieces2017}{Pecoraro et
al. 2017}), above 200 m approximately
(\citeproc{ref-sabarrosVerticalBehaviorHabitat2015}{Sabarros, Romanov,
and Bach 2015}).

This species has a high metabolic rate and, therefore, it requires large
energy supplies to fulfill the bioenergetics demands for movement,
growth, and reproduction
(\citeproc{ref-artetxe-arrateReviewFisheriesLife2021}{Artetxe-Arrate et
al. 2021}). Feeding behaviour is largely opportunistic, with a variety
of prey species being consumed, including large concentrations of
crustacean that have occurred recently in the tropical areas and small
mesopelagic fishes
(\citeproc{ref-krishnanDietCompositionFeeding2024}{Krishnan et al.
2024}; \citeproc{ref-rogerRelationshipsYellowfinSkipjack1994}{Roger
1994}; \citeproc{ref-duffyGlobalTrophicEcology2017}{Duffy et al. 2017}).
Recent growth studies have generally supported a two-stanza growth
curve, with a slow initial growth phase up to \(\sim\) 60 cm followed by
much faster growth (\citeproc{ref-farleyUpdatingEstimationAge2023}{J. H.
Farley et al. 2023}). In addition, differences in mean length-at-age
have been identified between males and females for fish older than four
years. Environmental variability in the IO impacts the abundance and
catch rates of this species. A significant negative association between
the Indian Ocean Dipoles (IODs) and the catch rates of yellowfin tuna
with a periodicity of approximately four years was observed
(\citeproc{ref-lanEffectsClimateVariability2013}{Lan, Evans, and Lee
2013}; \citeproc{ref-lanInfluenceOceanographicClimatic2020}{Lan, Chang,
and Wu 2020}). Likewise, Lan, Chang, and Wu
(\citeproc{ref-lanInfluenceOceanographicClimatic2020}{2020}) also found
that the El Niño Southern Oscillation (ENSO) had an impact on catch
rates near the Arabian Sea.

\subsection{Stock structure}\label{stock-structure}

Fisheries information indicates that adult yellowfin are distributed
continuously throughout the entire tropical Indian Ocean, but some more
detailed analysis of fisheries data suggests that the stock structure
may be more complex. The tag recoveries may indicate that the western
and eastern regions of the IO support relatively discrete
sub-populations of yellowfin tuna. Studies of stock structure using DNA
techniques have indicated that there may be genetically discrete
subpopulations of yellowfin tuna in the northwestern IO
(\citeproc{ref-dammannagodaEvidenceFineGeographical2008}{Dammannagoda,
Hurwood, and Mather 2008}) and within Indian waters
(\citeproc{ref-kunalMitochondrialDNAAnalysis2013}{Kunal et al. 2013}). A
recent study of stock structure using the gene sequencing technology
along with a basin-scale sampling design indicated genetic
differentiation between north and south of the equator within the IO,
and possibly additional genetic structure within the locations north of
the equator (\citeproc{ref-greweGeneticPopulationConnectivity2020}{Grewe
et al. 2020}). Parasite composition and abundance suggest limited
movement of yellowfin between Indonesian archipelago (eastern IO) and
the Maldives (central IO)
(\citeproc{ref-mooreMovementJuvenileTuna2019}{Moore et al. 2019}).
Isotope studies have also suggested relatively limited movement, with
resident behaviour at the temporal scale of their muscle turnover
(\(\sim\) 3 months)
(\citeproc{ref-menardIsotopicEvidenceDistinct2007}{Ménard et al. 2007}).
Otolith chemistry analyses concluded that fisheries operating in the
western IO are mainly composed of fish with western origin, which
suggest limited movement from east to west
(\citeproc{ref-artetxe-arrateOtolithStableIsotopesinreview}{Artetxe-Arrate
et al. in review}). These studies generally support the potential
presence of population units of yellowfin tuna within the IO, despite
that considerable uncertainty remains on sub-regional population
structure in this region. This assessment assumes that the IO yellowfin
tuna stock consists of several interconnected regional populations
(Figure~\ref{fig-4A-config}) that have the same biological
characteristics; however, we acknowledge that more studies are needed to
reveal the structure of this species.

\subsection{Fisheries}\label{fisheries}

Yellowfin tuna are harvested with a diverse variety of gear types, from
small-scale artisanal fisheries (in the Arabian Sea, Mozambique Channel
and waters around Indonesia, Sri Lanka, the Maldives, and Lakshadweep
Islands) to large gillnetters (from Oman, Iran and Pakistan operating
mostly but not exclusively in the Arabian Sea) and distant-water
longliners and purse seiners that operate widely in equatorial and
tropical waters (Figure~\ref{fig-catch-grid}). Purse seiners and
gillnetters catch a wide size range of yellowfin tuna, whereas the
longline fishery takes mostly adult fish (Figure~\ref{fig-agg-size}).

Prior to 1980, annual catches of yellowfin tuna remained below about
80,000 mt and were dominated by longline catches
(Figure~\ref{fig-catch-bar}). Annual catches increased markedly during
the 1980s and early 1990s, mainly due to the development of the
purse-seine fishery as well as an expansion of the other established
fisheries (fresh-tuna longline, gillnet, baitboat, handline and, to a
lesser extent, troll). A peak in catches was recorded in 1993, with
catches over 400,000 mt, the increase in catch almost fully attributable
to longline fleets, particularly longliners flagged in Taiwan, which
reported exceptional catches of yellowfin tuna in the Arabian Sea. The
Taiwanese longline fishery in the IO has been equipped with super-cold
storage. Since around 1986, the fleet has fished more frequently with
deep sets.

Catches declined in 1994, to about 350,000 mt, remaining at that level
for the next decade then increasing sharply to reach a peak of about
520,000 mt in 2004-2005 driven by a large increase in catch by all
fisheries, especially the purse-seine (free school) fishery. Total
annual catches declined sharply from 2004 to 2007 and remained at about
300,000 mt during 2007--2011. In 2012, total catches increased to about
400,000 mt and were maintained at about that level through 2013 to 2015.
Total catches increased to an average of 430,000 mt between 2016 and
2019, and a maximum of close to 450,000 mt in 2019
(Figure~\ref{fig-catch-bar}), despite IOTC Resolution 17/01 which
requested major fleets to substantially reduce their yellowfin catches
below the 2014 or 2015 catch level. Furthermore, catch levels of about
440,000 mt reported for 2018 might be under-estimated (to some extent)
because of changes in data processing methodology by European
Union-Spain for its purse seine fleet for that year
(\citeproc{ref-iotcReviewYellowfinTuna2021}{IOTC 2021}).

In recent years (2015--2023), purse seine has been the dominant fishing
method harvesting 36\% of the total IO yellowfin tuna catch (by weight),
with the gillnet and handline fisheries, comprising 20\% and 18\% of the
catch, respectively. There was a substantial increase in the catch by
handline in 2020 (Figure~\ref{fig-catch-bar}). A smaller component of
the catch was taken by industrial longline (5\%), and the regionally
important baitboat (4\%) and troll (4\%) fisheries. The recent increase
in the total catch has been mostly attributable to an increase in catch
from the gillnet and handline fisheries.

The purse-seine catch is generally distributed equally between
free-school and associated (log and FAD sets) schools, although the
large catches in 2003--2005 were dominated by fishing on free-schools.
Conversely, during 2015--2023 the purse-seine catch was dominated (70\%)
by the associated fishery.

Historically, most of the yellowfin catch has been taken from the
western equatorial region of the IO (44\%; region 1b,
Figure~\ref{fig-catch-grid}) and, to a lesser extent, the Arabian Sea
(26\%), the eastern equatorial region (24\%, region 4) and the
Mozambique Channel (5\%; region 2). The purse-seine and baitboat
fisheries operate almost exclusively within the western equatorial
region, while catches from the Arabian Sea are principally by handline,
gillnet, and longline (see Figure~\ref{fig-catch-grid}). Catches from
the eastern equatorial region (region 4) were dominated by longline and
gillnet (around Sri Lanka and Indonesia). The southern IO (region 3)
accounts for a small proportion of the total yellowfin catch (1\%) taken
exclusively by longline.

In recent years (2008--2012), due to the threat of piracy, the bulk of
the industrial purse seine and longline fleets moved out of the western
waters of Region 1b to avoid the coastal and off-shore waters off
Somalia, Kenya and Tanzania. The threat of piracy particularly affected
the freezer longline fleet and levels of effort and catch decreased
markedly from 2007. The total catch by freezing longliners declined to
about 2,000 mt in 2010, a 10-fold decrease in catch from the years
before the onset of piracy. Purse seine catches also dropped in
2007--2009 and then started to recover. Piracy off the Somali coast was
almost eliminated by 2013 but longline catches have not recovered.

The sizes caught in the IO range from 30 cm to 180 cm fork length
(Figure~\ref{fig-agg-size}). Intermediate age yellowfin are seldom taken
in the industrial fisheries, but are abundant in some artisanal
fisheries, mainly in the Arabian Sea. Newly recruited fish are primarily
caught by the purse seine fishery on floating objects and the
pole-and-line fishery in the Maldives. Males are predominant in the
catches of larger fish at sizes larger than 150 cm (this is also the
case in other oceans). Medium sized yellowfin concentrate for feeding in
the Arabian Sea.

\section{Model structure}\label{model-structure}

\subsection{Spatial stratification}\label{spatial-stratification}

The geographic area considered in the assessment is the IO, defined by
the coordinates 40\(^\circ\)S-25\(^\circ\)N and
20\(^\circ\)E-150\(^\circ\)E. Earlier yellowfin stock assessments have
adopted a five-area spatial structure
(\citeproc{ref-langleyStockAssessmentYellowfin2012}{Langley, Herrera,
and Million 2012}), but several issues were identified for that
structure. Since 2015, a four-area spatial structure is used for this
stock (Figure~\ref{fig-4A-config}). The Arabian Sea (area 1a) and
western equatorial region (area 1b) make up the model area 1 but kept
the fishery information separated (i.e., areas-as-fleet approach) to
account for differences in selectivity between these sub-areas
(\citeproc{ref-puntSpatialStockAssessment2019}{Punt 2019}). The spatial
structure retains two regions that encompass the main year-round
fisheries in the tropical area (areas 1 and 4) and two austral,
subtropical regions where the longline fisheries occur more seasonally
(areas 2 and 3).

The current spatial structure separates the purse-seine fishery in the
northern Mozambique Channel (10-15\(^\circ\)S) from the equatorial
region, as the fishery in the northern Mozambique Channel exhibits
strong seasonal variation in effort and operates differently from the
equatorial region
(\citeproc{ref-langleyStockAssessmentYellowfin2015}{Langley 2015}).
There is also a separation of the purse-seine fishery between the
western and eastern tropical region with the current boundary between
region 1b and region 4. In addition to the four-area configuration, we
also evaluated two more spatial structures: one-area and two-area
configurations (see more details in Appendix XX). The 2021 assessment
also evaluated a modified version of the four-area structure
(\citeproc{ref-fuPreliminaryIndianYellowfin2021}{Fu et al. 2021}), but
there were some constraints to evaluate that configuration in the
current assessment (see Section XX).

\subsection{Temporal stratification}\label{temporal-stratification}

The time period covered by the assessment is 1950−2023, which represents
the period for which catch data are available from the commercial
fishing fleets. Langley
(\citeproc{ref-langleyStockAssessmentYellowfin2015}{2015}) suggested
that the assessment results were not sensitive to the early catches from
the model (pre-1972) and commencing the model in 1950 or 1972 (assuming
unexploited equilibrium conditions) yielded very similar results.

The time step in the assessment model was quarter (i.e., three months
duration, four quarters per year), representing a total of 296 model
time steps. The definition of these time steps enabled recruitment to be
estimated for each quarter to approximate the continuous recruitment of
yellowfin in the equatorial regions. In addition, the quarterly model
time step precluded the estimation of seasonal model parameters,
particularly the movement parameters. Fu et al.
(\citeproc{ref-fuPreliminaryIndianOcean2018}{2018}) explored an
alternative annual/seasonal model structure which explicitly estimated
seasonal movement dynamics. However, the alternative temporal structure
did not yield substantially different results.

\section{Model inputs}\label{model-inputs}

Catch (1950-2023) and size (1952-2023) information was provided by the
IOTC Secretariat in a comma-separated values (CSV) format. These
datasets and the metadata can be found online at the IOTC website:
\url{https://iotc.org/documents/WPTT/26AS/Data/01}. Four indices of
abundance were also available: joint longline, purse seine fishing on
free schools, purse seine fishing on associated schools, and the
associative behavior-based abundance index. These indices can also be
found online at:
\url{https://iotc.org/documents/standardised-cpue-yft-and-bet}. Release
and recovery data (2005-2014) from two tagging programs were also
available, as well as age data from the GERUNDIO project (2013-2021).

\subsection{Definition of fisheries}\label{definition-of-fisheries}

The assessment adopted the equivalent fisheries definitions used in the
previous stock assessments (Table~\ref{tbl-fishery-codes}). First, nine
\emph{fishery groups} were defined based on fleet, gear, purse seine set
type, and type of vessel in the case of longline fleet
(Table~\ref{tbl-fishery-codes}), which represent relatively homogeneous
fishing units, with similar selectivity and catchability characteristics
that do not vary greatly over time. Then, \emph{fishery groups} were
divided into regions (Table~\ref{tbl-fleet-4A}), producing a total of
twenty-one \emph{fisheries} in the assessment model. A brief description
of each \emph{fishery group} is provided below.

The longline fishery was partitioned into two main components:

\begin{itemize}
\item
  \emph{Freezing longline fisheries (LL)}, or all those using drifting
  longlines for which one or more of the following three conditions
  apply: (i) the vessel hull is made up of steel; (ii) vessel length
  overall of 30 m or greater; (iii) the majority of the catches of
  target species are preserved frozen or deep-frozen. A composite
  longline fishery was defined in each model area aggregating the
  longline catch from all freezing longline fleets (principally Japan
  and Taiwan).
\item
  \emph{Fresh-tuna longline fisheries (LF)}, or all those using drifting
  longlines and made of vessels (i) having fibreglass, fibre reinforced
  plastic, or wooden hull; (ii) having length overall less than 30 m;
  (iii) preserving the catches of target species fresh or in
  refrigerated seawater. A composite longline fishery was defined
  aggregating the longline catch from all fresh-tuna longline fleets
  (principally Indonesia and Taiwan) in region 4, which is where the
  majority of the fresh-tuna longliners have traditionally operated.
\item
  The purse-seine catch and effort data were apportioned into two
  separate method fisheries: catches from sets on associated schools of
  tuna (log and drifting FAD sets; \emph{LS}) and from sets on
  unassociated schools (free schools; \emph{FS}).
\item
  A single baitboat fishery (\emph{BB}) was defined within region 1b
  (essentially the Maldives fishery).
\item
  Gillnet fisheries (\emph{GI}) were defined in the Arabian Sea (region
  1a), including catches by Iran, Pakistan, and Oman, and in region 4
  (Sri Lanka and Indonesia).
\item
  Three troll fisheries (\emph{TR}) were defined, representing separate
  fisheries in regions 1b (Maldives), 2 (Comoros and Madagascar) and 4
  (Sri Lanka and Indonesia).
\item
  A handline fishery (\emph{HD}) was defined within region 1a,
  principally representing catches by the Yemeni fleet.
\item
  A miscellaneous ``Other'' fishery (\emph{OT}) was defined comprising
  catches from artisanal fisheries other than those specified above
  (e.g.~trawlers, small purse seines or seine nets, sport fishing, and a
  range of small gears).
\end{itemize}

\subsection{Catch}\label{catch}

The catch datasets was composed of information about time (year and
month), fleet, gear type, type of association of the fish school, grid
code at a \(5^\circ\times 5^\circ\) resolution, and catch in weight
(metric tons) and numbers. The grid code contained information on the
grid resolution, quadrant, and longitude and latitude of the corner of
the grid.

\subsubsection{Processing}\label{processing}

We followed the next steps to produce the catch input for SS3:

\begin{itemize}
\tightlist
\item
  Month information was used to assign quarters (i.e., four quarters
  from January to December).
\item
  Fishery group was assigned based on the fleet, gear type, and the type
  of association of the fish school.
\item
  Catch was summed by year, quarter, fleet, gear type, fishery group,
  and grid code.
\item
  The grid code was used to calculate the longitude and latitude of the
  center of the grid (called \emph{centroid} hereafter).
\item
  The centroid was used to assign regions used in the assessment model
  (Figure~\ref{fig-map-grid-agg2}).
\item
  Fishery was assigned based on fishery group and region.
\end{itemize}

\subsubsection{Reassignment}\label{sec-catchreassign}

To simplify the fishery structure in the stock assessment model, we
reassigned catches in areas with low fishing activity to main areas as
done in the 2021 assessment.

\begin{itemize}
\item
  \emph{LF} fisheries: catch in regions 1 to 3, representing only
  \(\sim\) 3\% of the total catches over the time series, were assigned
  to region 4.
\item
  \emph{FS} and \emph{LS} fisheries: purse seine catches in region 1a
  and 3 were reassigned to region 1b and 4, respectively.
\item
  \emph{BB} fisheries: a small proportion of the total baitboat catch
  and effort occurs on the periphery of region 1b, within regions 1a and
  4. Therefore, we assigned all \emph{BB} catches to region 1b.
\item
  \emph{GI} fisheries: a very small proportion of the total gillnet
  catch and effort occurs in regions 1b and 2, which was reassigned to
  area 1a. Likewise, catch in region 3 was reassigned to region 4.
\item
  \emph{TR} fisheries: moderate troll catches are taken in regions 1a
  and 3, which were reassigned to regions 1b and 4, respectively.
\item
  \emph{HD} fisheries: moderate handline catches are taken in regions
  1b, 2 and 4, which were reassigned to region 1a.
\item
  \emph{OT} fisheries: catch from region 1b and 2 was reassigned to
  region 1a, while catch from region 3 was reassigned to region 4.
\end{itemize}

\subsubsection{Aggregation}\label{aggregation}

After catch reassignment, catch data (in metric tons) was summed by
year, quarter, and fishery, and then organized in a SS3 format assuming
an error of 0.01. Overall, the time series of catches were quite similar
to the catch series included in the 2021 assessment
(Figure~\ref{fig-comp-catch}). The largest differences were observed for
\emph{OT} and \emph{TR} in region 4, especially during the last two
decades. Also, current catch estimates for \emph{LL} in region 3 are
slightly larger than the previous assessment during 2008-2020. The
changes are mostly attributed to revisions of catch estimation by the
IOTC Secretariat.

\subsection{Size data}\label{size-data}

The size data was composed of information about time (year and month),
fleet, gear type, type of association of the fish school, grid code,
number of fish sampled per fork length bin (cm), and the score of
reporting quality (RQ)
(\citeproc{ref-herreraProposalSystemAssess2010}{Herrera 2010};
\citeproc{ref-iotcReviewStatisticalData2024}{IOTC 2024}). The length bin
width was 2 cm and the length bins spanned from 10 to 340 cm. The data
were collected from a variety of sampling programmes, which can be
summarized as follows:

\begin{itemize}
\item
  \emph{FS} and \emph{LS} fisheries: Length-frequency samples from purse
  seiners have been collected from a variety of port sampling programmes
  since the mid-1980s. The samples are comprised of very large numbers
  of individual fish measurements. The length frequency samples are
  available by set type with sets catches from associated sets typically
  composed of smaller fish than free school catches
  (Figure~\ref{fig-agg-size}). The size composition of the catch from
  the free-school fishery is bimodal, being comprised of the smaller
  size range of yellowfin and a broad mode of larger fish
  (Figure~\ref{fig-agg-size}). The bimodal distribution is likely to
  have reflected different types of schools in the catch composition
  (e.g., free schools of mostly large adult yellowfin, or mixed species
  schools consisting of smaller yellowfin, M. Chassot, pers. comm.).
  Hence, the relative composition of large (\textgreater80cm) vs.~small
  (\textless80cm) yellowfin in the purse seine free schools fluctuates
  considerably over time. Between 2010 and 2020, there was both a dip in
  the average size of large fish caught in the FAD fishery, and a
  temporary increase in the average sizes of large fish caught in the
  free school fishery (Figure~\ref{fig-mlen}). There is also
  considerable catch of smaller fish taken during free school fishing
  operation in the Mozambique Channel area in region 2
  (\citeproc{ref-chassotAreThereSmall2014}{Chassot 2014}). The
  free-school fishery in region 4 appears to catch larger fish
  (Figure~\ref{fig-size-grid}).
\item
  \emph{LL} fishery: Length and weight data have been collected from
  sampling at ports and aboard Japanese commercial, research vessels,
  and observer programmes. Weight frequency data collected from the
  fleet have been converted to length frequency data via a processed
  weight-whole weight conversion factor and a weight-length key. Length
  frequency data from the Taiwanese longline fleet from 1980−2003 were
  included in the 2018 assessment, although data from the more recent
  years were excluded due to concerns regarding their reliability
  (\citeproc{ref-greehanReviewLengthFrequency2013}{Greehan and Hoyle
  2013}). Length data are also available from other fleets (e.g.,
  Seychelles, Korean, China, etc.) in more recent years. Analyses of
  size data show that the average lengths of yellowfin caught by the
  longline fleet are generally larger in the southern regions,
  particularly in the southwest
  (\citeproc{ref-hoyleReviewSizeData2021}{Hoyle 2021}). There is
  considerable temporal variation in the length of fish caught
  (Figure~\ref{fig-mlen}), but some of this variation is inconsistent
  between datasets, such as temporal patterns of variation in the 1970s
  that differ between length and weight data from the Japanese fleet.
  For all longline fisheries there was a marked decline in the size of
  fish caught by Japan during the 1950s and 1960s, while the size of
  fish caught stabilised during the 1970s and 1980s
  (Figure~\ref{fig-mlen}).
\item
  \emph{LF} fishery: Length and weight data were collected in port,
  during unloading of catches, for several landing locations and time
  periods, especially on fresh-tuna longline vessels flagged in
  Indonesia and Taiwan/China (IOTC-OFCF sampling).
\item
  \emph{GI} fishery: Samples come from Iran, Pakistan, Sri Lanka, and
  Oman in the Arabian Sea from 1987, and from Indonesia and Sri Lanka in
  other tropical areas from 1975.
\item
  \emph{BB} fishery: Size data come principally from the Maldivian fleet
  from 1983 with a large proportion of juveniles. Also, samples from
  Indonesia and Sri Lanka are also available for some years but with a
  low sample size.
\item
  \emph{TR} fishery: Samples come mainly from Comoros in the western IO
  from 2015, although some small samples are also available from EU
  (France, Mayotte) and Maldives. In the eastern IO, size data come from
  Indonesia (1985-1990 and after 2019) and Sri Lanka (1994-2018) fleets.
\item
  \emph{HD} fishery: Samples come exclusively from the Arabian Sea
  region and from a high diversity of fleets, although the Maldivian
  fleet has been the most consistent over the years (1985-2023). Limited
  sampling was conducted over the last decade.
\item
  \emph{OT} fishery: Samples are available from 1983 in the eastern IO
  and from 1997 in the western IO. The main fleets are the Indonesian,
  Sri Lankan, Maldivian, and Mozambique. Limited samples are available
  during the last few years.
\end{itemize}

The IOTC Secretariat provided two size datasets with two distinct grid
resolutions:

\begin{itemize}
\tightlist
\item
  \emph{Original data}: the size dataset had six main types of grid
  dimensions (see Table~\ref{tbl-grid-size} and
  Figure~\ref{fig-map-grid-agg1}), although \(\sim\) 97\% of
  observations were category 5 or 6. A seventh category was also present
  that covered the Seychelles National Jurisdiction Area, but those
  observations were removed from the size database. This type of size
  dataset was used in the 2021 assessment.
\item
  \emph{cwp55 data}: the size dataset was provided at a
  \(5^\circ\times 5^\circ\) grid resolution
  (Figure~\ref{fig-map-grid-agg2}).
\end{itemize}

\subsubsection{Processing}\label{processing-1}

First, we removed gear types with unclear classification (\emph{HOOK},
\emph{HATR}, \emph{PSOB}, and \emph{PS} with unclassified school type
\emph{UNCL}). Then, we reduced the number of length bins in the data by
summing the number of sampled fish equal or larger than 198 cm and
assign it to the 198 cm length bin. Likewise, we summed the number of
sampled fish \(\leq\) 10 cm and assigned to the 10 cm length bin. We
followed these steps to produce the SS3 size inputs:

\begin{itemize}
\tightlist
\item
  Month information was used to assign quarters, having four quarters
  from January to December.
\item
  Fishery group was assigned based on the fleet, gear type, and type of
  association of the fish school.
\item
  The number of sampled fish per length bin was summed and the RQ was
  averaged by year, quarter, fleet, gear type, fishery group, and grid
  code.
\item
  The grid code was used to calculate the grid centroid.
\item
  The centroid was used to assign regions. Note that this region
  assignment varied depending on the type of dataset (see
  Figure~\ref{fig-map-grid-agg1} and Figure~\ref{fig-map-grid-agg2}).
\item
  Fishery was assigned based on fishery group and region.
\item
  We converted the length bin width from 2 to 4 cm. To do so, we summed
  the number of sampled fish from pairs of length bins (e.g., 10 and 12
  cm were summed and assigned to 10 cm, 14 and 16 cm were summed and
  assigned to 14 cm, and so on). After this conversion, we had a total
  of 48 length bins.
\end{itemize}

Only for the size dataset with regular grids, we then assigned the catch
(in numbers) that corresponded to every observation in the size data
(i.e., year, quarter, grid, fleet, and gear type). We found a perfect
match for \(\sim\) 74\% of cases but there were some size observations
without catch. In order to fill in these catch gaps, we followed an
imputation procedure with five levels:

\begin{itemize}
\tightlist
\item
  \emph{Level 1}: Fill in catch gaps with the average catch per grid for
  a given year, quarter, fleet, gear type, and fishery group.
\item
  \emph{Level 2}: Fill in catch gaps with the average catch per grid for
  a given year, fleet, gear type, and fishery group.
\item
  \emph{Level 3}: Fill in catch gaps with the average catch per grid for
  a given year, gear type, and fishery group.
\item
  \emph{Level 4}: Fill in catch gaps with the average catch per grid for
  a given gear type and fishery group.
\item
  \emph{Level 5}: Fill in catch gaps with the average catch per grid for
  a given fishery group.
\end{itemize}

Figure~\ref{fig-imputation} shows the percentage of size observations
that needed each level of imputation.

\subsubsection{Reassignment}\label{reassignment}

We conducted the size data reassignment as done for the catch data in
Section~\ref{sec-catchreassign}.

\subsubsection{Filtering}\label{filtering}

In order to remove inconsistent patterns in the length frequency data,
we carried out these filters:

\begin{itemize}
\item
  The first filter was to remove observations with less than 100 fish
  sampled and not considered best quality based on the RQ score.
\item
  \emph{LL} fishery: a review of the longline size data shows that the
  sampling behaviour of Taiwanese and Seychelles fleets (mostly
  reflagged Taiwanese vessels) have changed over time, with patterns in
  the logbook length data inconsistent with other fleet
  (\citeproc{ref-hoyleReviewSizeData2021}{Hoyle 2021}), and as such the
  WPTT23 (Data Preparatory) recommended omitting all Taiwanese and
  Seychelles logbook length data from the 2021 assessment
  (\citeproc{ref-iotcReviewYellowfinTuna2021}{IOTC 2021}). Following
  this advice, we removed length frequency data from the Taiwanese and
  Seychelles longline logbooks from the final length frequency data
  sets.
\item
  \emph{LL} fishery: longline length frequency data during 1970-1995 and
  2010-2020 in region 1a was removed.
\item
  \emph{LL} fishery: attempts to fit the size data of this fishery in
  past assessments suggested that the large decline in mean size
  observed before 1960 is inconsistent with the yellowfin population
  dynamics. Hoyle (\citeproc{ref-hoyleReviewSizeData2021}{2021}) suggest
  that selectivity may have changed during this early period and
  recommend avoiding fitting to these data with the same selectivity.
  Therefore, we omitted longline size data before 1960 for regions 1b,
  2, 3, and 4.
\item
  \emph{LL} fishery: longline length frequency data in 2001-2005, 2015,
  and 2019 in region 4 was removed.
\item
  \emph{LF} fishery: we removed size data before 2005.
\item
  \emph{GI} fishery: we removed size data from the Sri Lankan fleet in
  2021.
\item
  \emph{HD} fishery: we removed size data from the Maldivian fleet in
  2003.
\item
  \emph{OT} fishery: we removed size data sampled during 2021-2022 in
  region 1a, and data sampled in 2016 in region 4.
\item
  \emph{TR} fishery: we removed size data in region 1b and 2. Also, we
  removed size data in region 4 from 2016 to 2019.
\end{itemize}

The filters applied to the \emph{LL}, \emph{LF}, and \emph{TR} fisheries
were also applied in the 2021 assessment. Other filters were exclusive
to the current assessment.

\subsubsection{Aggregation}\label{aggregation-1}

After filtering and in order to aggregate the size data by year,
quarter, and fishery for SS3, we followed two approaches: simple and
catch-raised aggregation.

\begin{itemize}
\item
  \emph{Simple aggregation}: this type of aggregation was performed only
  for the \emph{original} size dataset. We summed the number of sampled
  fish per length bin and averaged the RQ values by year, quarter, and
  fishery. RQ values was treated as input sample size in SS3, but we
  also tested using an input sample size of 5 for every observations as
  done in the 2021 assessment. This aggregation approach was used in the
  2021 assessment and assumes that the collection of samples was broadly
  representative of the operation of the fishery in each quarter.
\item
  \emph{Catch-raised aggregation}: this type of aggregation was
  performed only for the \emph{cwp55} size dataset. We performed a
  catch-weighted sum of the number of sampled fish by length bin and a
  catch-weighted average of the RQ values. RQ values was treated as
  input sample size in SS3.
\end{itemize}

A graphical representation of the availability of length samples is
provided in Figure~\ref{fig-rq-size}. The longline fishery provided size
data from the 1960s but with relatively low quality during the first
decades. Most of the size information started to appear in the 1980s,
being the purse seine fisheries the most consistent and with the highest
quality (Figure~\ref{fig-rq-size}). The RQ scores did not differ between
aggregation methods.

The differences in size compositions between the simple and catch-raised
aggregation methods were minimal for most fisheries
(Figure~\ref{fig-agg-size}). The largest differences were found for the
\emph{OT} fisheries in region 1a and 4, and for the handline fishery in
region 1a. We observed an increase in the mean length for the \emph{LL}
fisheries in all regions (Figure~\ref{fig-mlen}). In the case of the
free school purse seine fishery, we also noted an increase in mean
length over time in region 1b. Conversely, we noted a decrease in the
mean length for the log school purse seine fishery over the years in
regions 1b, 2, and 4, especially from 1980 to \(\sim\) 2005. For the
handline fishery, we noted an increase in mean length from the 1990
until \(\sim\) 2010, and then a decrease until recent years. These
patterns were quite similar between the two aggregation methods
(Figure~\ref{fig-mlen}).

The size compositions used in the current assessment were comparable
with the 2021 assessment (both using the simple aggregation approach),
although small differences can be observed for the \emph{OT} and
\emph{TR} fisheries in region 4 (Figure~\ref{fig-comp-agg-size}).
Regarding mean length, most fisheries other than longline had similar
tendencies over time when comparing 2021 and current values. For
longline \emph{LL} fisheries, size compositions in 2021 had larger mean
lengths before 1990 compared with the current size compositions
(Figure~\ref{fig-comp-mlen}).

\subsection{Indices}\label{indices}

\subsubsection{Longline CPUE}\label{longline-cpue}

Standardised CPUE indices (1975-2023) were derived using a hurdle
generalized linear model (GLM) from longline catch and effort
information provided by Japan, Korea, and Taiwan
(\citeproc{ref-matsumotoJointLonglineCPUE2024}{Matsumoto et al. 2024}).
The data used for the standardization included operation date, fishing
location, vessel ID, fishing effort (number of hooks per set), and catch
in numbers. Cluster analyses of species composition data for each fleet
and model area were used to separate datasets into fisheries understood
to target different species. Selected clusters were then combined and
standardized using GLMs. The log-transformed yellowfin catch per number
of hooks set was the dependent variable of the positive model component,
while the probability of catch rate being zero was the dependent
variable in the binomial model component. In addition to the year and
quarter variables, GLMs included covariates for 5\(^\circ\) square
location, cluster, and vessels ID. The data used in the GLM was
subsampled from the operational data (\$\sim\(10-30%) or aggregated by 5
\)\^{}\circ \times\$ 5\(^\circ\) grid. Moreover, data from regions 1a
and 1b were combined as a single region. The CPUE indices was produced
at an annual and quarterly time step and for each model area.

The CPUE index produced for region 1 was assigned to area 1b in the
stock assessment model. For the regional longline fisheries, a common
catchability coefficient (and selectivity) was estimated in the
assessment model, thereby, linking the respective CPUE indices among
regions. This significantly increases the power of the model to estimate
the relative (and absolute) level of biomass among regions. However, as
CPUE indices are essentially density estimates it is necessary to scale
the CPUE indices to account for the relative abundance of the stock
among regions. For example, a relatively small region with a very high
average catch rate may have a lower level of total biomass than a large
region with a moderate level of CPUE.

We determined regional scaling factors that incorporated both the size
of the region and the relative catch rate to estimate the relative level
of exploitable longline biomass among regions. This approach was also
used in the 2021 assessment, and is similar to that used in the Western
and Central Pacific Fisheries Commission (WCPFC) regionally
disaggregated tuna assessments. Hoyle
(\citeproc{ref-hoyleIndianOceanTropical2018}{2018}) proposed a set of
regional weighing factors for IO yellowfin based on aggregated longline
catch effort data. The authors recommended the estimates by method `8'
for the period 1979--1994 (referred to as `7994m8', see Table 2 of Hoyle
(\citeproc{ref-hoyleIndianOceanTropical2018}{2018})) to be included in
the current assessment. The relative scaling factors calculated for
regions 1--4 are 1.674, 0.623, 0.455 and 1, respectively.

For each of the principal longline fisheries, the standardized CPUE
index was normalized to the mean of the period for which the region
scaling factors were derived (i.e., the GLM index from 1979--1994). The
normalized GLM index was then scaled by the respective regional scaling
factor to account for the regional differences in the relative level of
exploitable longline biomass among regions (Figure~\ref{fig-ts-cpue}).

A number of important trends are evident in the CPUE indices:

\begin{itemize}
\item
  The western tropical (region 1b) CPUE increased during the late 1970s
  and early 1980s, then suddenly declined from 1987 to 1990. After 1990,
  the CPUE in this region remained roughly stable until the late 2000s,
  that coincided with a number of piracy incidents in the western Indian
  Ocean (2008--2011). After that time, it remained close to the lowest
  level observed in that region but showed very large seasonal and
  annual variations. From 2020, we noticed a remarkable increase in CPUE
  values compared to the 2013-2019 period.
\item
  The eastern tropical region 4 followed a similar pattern until 1990
  but then declined steadily, and by 2016 was also close to the lowest
  level in the time series. The recent decline in CPUE in this region is
  consistent with a decline in the proportion of yellowfin in the
  combined tuna catch from the Japanese longline fleet in the eastern
  IO. It is unclear whether the change in species proportion is related
  to a decline in the abundance of yellowfin in the region (relative to
  the other species) or a regional change in the targeting of the
  fishing fleet. However, there is an indication that there has been a
  differential shift towards deeper longline gear (greater HBF) in the
  eastern IO since 2000 and this may indicate a shift in targeting
  toward bigeye tuna in this region (Hoyle pers. comm. additional JP LL
  analyses). Such factors may not be adequately accounted for in the
  standardisation of the yellowfin CPUE data.
\item
  The CPUE indices in western temperate region 2 followed a similar
  pattern to the western tropical indices, with a decline until the
  late-1970s followed by an increase until the late 1980s, and
  subsequently a slow decline with significant variability. From 1990,
  the CPUE in this region has remained roughly stable, with no
  remarkable increase during the last years.
\item
  The CPUE indices from region 3 are low compared to the other three
  regions reflecting the low regional scaling factor. However, the
  overall trend in the CPUE indices is broadly comparable to the other
  regions. The eastern temperate region 3 the pattern was similar to the
  western temperate area before 1979. After 1979 catch rates increased
  until the mid-2000's, but then declined rapidly and reached their
  lowest observed levels by 2016. In this region, we also noticed an
  increase in CPUE for the last years.
\item
  There is an exceptionally high peak in CPUE indices 1976--78, which is
  also associated with a high uncertainty. Hoyle, Satoh, and Matsumoto
  (\citeproc{ref-hoyleExploringPossibleCauses2017}{2017}) showed that
  this discontinuity exists in Japanese, Taiwanese and Korean data, and
  in multiple regions in multiple oceans, and for both bigeye and
  yellowfin tuna. Hoyle, Satoh, and Matsumoto
  (\citeproc{ref-hoyleExploringPossibleCauses2017}{2017}) suggested this
  is unlikely to be explained by changes to the population or
  catchability but may be associated with catch reporting and data
  management.
\item
  The spike in the CPUE indices around 2012 in the west equatorial
  region (region 1) was evident for most fishing fleets. Several
  hypotheses have been proposed on what could have caused CPUE to have
  increased, including a return to fishing in areas that were most
  affected by piracy. However, further investigation is required.
\end{itemize}

The values and trend of LL CPUE used in the 2021 assessment and in this
assessment were quite similar for all model areas before 1990
(Figure~\ref{fig-comp-cpue}). After 1990, we noted large differences for
model area 1b, where the current CPUE showed consistently larger values
( \(\sim\) 40\%) than the 2021 CPUE, especially after 2005. For region
4, we also observed slightly larger 2024 CPUE values compared to the
2021 CPUE after 2005. However, this discrepancy was minimal for regions
2 and 3.

\subsubsection{Purse seine CPUE indices}\label{purse-seine-cpue-indices}

The European and associated flags purse seine fishing activities in the
Indian Ocean during 1981--2020 have been monitored through the
collection of logbooks and observer sampling. Standardised indices of
the biomass of yellowfin caught by European purse seiners (Spain and
France) from sets on free swimming schools (1991 -- 2022) and sets on
associated tuna schools (FOB, 2010 -- 2022) were developed
(Figure~\ref{fig-ts-ps-cpue}). The free school index was based on the
application of a general additive mixed effect model with three
components to model
(\citeproc{ref-kaplanStandardizedCPUEAbundance2024}{Kaplan et al.
2024}): i) the detection rate of free swimming schools per unit search
time, ii) the probability that adult yellowfin are present in a set, and
iii) the adult biomass per set given presence assuming a lognormal
distribution. The FOB index was based on the application of two models:
a generalized linear mixed model and a spatiotemporal model, both using
a hurdle approach (\citeproc{ref-correaStandardizedCatchUnit2024}{Correa
et al. 2024}). These standardizations considered a comprehensive list of
candidate covariates, including the effect of the technological
improvement related to the use of echosounder buoys and environmental
variables. The predicted CPUE over time was obtained using the
\emph{predict-then-aggregate} approach, which is considered best
practice (\citeproc{ref-hoyleCatchUnitEffort2024}{Hoyle et al. 2024}).

The FOB mainly informs the biomass of juvenile yellowfin, while the
free-school index informs the biomass of the adult portion of the
population in region 1b. The FOB index displays juvenile biomass
fluctuations over the years, with larger values during 2013 and 2014,
and a remarkable increase after 2020. On the other hand, the free school
CPUE index showed a increase from the late 1990s until 2004, and then a
dramatic decrease until 2009. From 2015, the free school index showed
another dramatic decrease until 2018, and a slight recovery after that.

We evaluated the impact of incorporating the free school and FOB CPUE
series as auxiliary indices (i.e., always in conjunction with the LL
CPUE index) in the assessment model. In order to incorporate the free
school index, we revised the region/fishery structure in which purse
seine fisheries are separated into the small and large fish components
(EXPAND THIS).

\subsubsection{Effort creep}\label{effort-creep}

It is well recognised that the relationship between PS CPUE and
abundance is unlikely be proportional, as the improvement of catch
efficiency due to technology development is difficult to quantify, and
the changes in catchability are not fully accounted for in the
standardisation process. Effort creep can be defined as an unquantified
increase in the average fishing power over time that disturbs the
relationship of proportionality between the index and the stock
trajectory (\citeproc{ref-hoyleEffortCreepLongline2024}{Hoyle 2024}).
These changes in catchability over time can affect CPUE indices, and
therefore the outcomes of stock assessments. This is especially
important for assessments that lack abundance indices from
fishery-independent surveys, which includes the majority of the
fisheries managed by tuna regional fishery management organizations
(RFMOs).

For the case of longline fleets, technological advances include
electronic devices to help navigate, communicate, and find target
species. Synthetic materials allowed fishers to improve hooks and lines
which increased probabilities of both hooking and landing. Satellite
imagery improved search efficiency. Freezers increased the proportion of
time spent on fishing grounds, while equipment for faster longline
retrieval increased hooks set without affecting soak time
(\citeproc{ref-hoyleEffortCreepLongline2024}{Hoyle 2024}). In the 2021
yellowfin assessment in the IO, a sensitivity analysis was run during
the WPTT meeting that included 1\% effort creep per year for the entire
period of the index, which resulted in changes in the stock depletion
level. Based on the recommendations of Hoyle
(\citeproc{ref-hoyleEffortCreepLongline2024}{2024}), we evaluated two
different levels of effort creep for the LL CPUE index in the current
assessment: 0.5 and 1.5\% per annum, which is associated with vessel
turnover.

The WCPFC assessments have often estimated substantial changes in PS
FAD-associated fisheries (e.g., McKechnie, Pilling, and Hampton
(\citeproc{ref-mckechnieStockAssessmentBigeye2017}{2017})). Using a
similar approach, Kolody
(\citeproc{ref-kolodyEstimationIndianOcean2018}{2018}) estimated a
catchability increase of approximately 1.25\% per year for the
standardised purse seine effort for yellowfin from sets on associated
schools. Likewise, studies on the French fleet indicate a 10\% increase
in catch per set associated with echosounder use, equivalent to about
1\% per annum, and a 1.7 -- 4.0 \% increase in efficiency (stable across
time) arising from fishing their own floating objects
(\citeproc{ref-wainQuantifyingIncreaseFishing2021}{Wain et al. 2021}).
Based on the recommendations of Hoyle
(\citeproc{ref-hoyleEffortCreepLongline2024}{2024}), we evaluated two
different levels of effort creep for both purse seine indices in the
current assessment: 1.5 and 4.35\% per annum.

\subsection{Age data}\label{age-data}

Age and size information was available for fish sampled between 2009 and
2022 (Figure~\ref{fig-caal-nsamp}) from the \emph{GERUNDIO} project that
aimed the collection and analysis of biological samples of tropical
tunas, swordfish, and blue sharks to improve age, growth, and
reproduction data for the IOTC. In a first step, J. Farley et al.
(\citeproc{ref-farleyEstimatingAgeGrowth2021}{2021}) presented this data
that contained information from otoliths from 253 yellowfin tuna sampled
mainly in the western IO. Then, J. H. Farley et al.
(\citeproc{ref-farleyUpdatingEstimationAge2023}{2023}) presented an
updated dataset with age and length information from additional 136
individuals. To calculate decimal age of sampled fish, daily and annual
aging methods were used. Decimal age was calculated for each fish with
an annual count based on the method developed for yellowfin and bigeye
tuna (\emph{Thunnus obesus}) in the western Pacific Ocean
(\citeproc{ref-farleyAgeGrowthYellowfin2020}{J. Farley et al. 2020}). To
find more details on the ageing estimation method, see J. H. Farley et
al. (\citeproc{ref-farleyUpdatingEstimationAge2023}{2023}).

Age and fork length information from a total of 389 individuals was
provided to be used in the current stock assessment. This source of
information can be included in SS3 as conditional age-at-length data,
which is important to inform growth and stock age structure
(\citeproc{ref-leeUseConditionalAge2019}{Lee et al. 2019}). We followed
these steps to produce the input CAAL data for SS3:

\begin{itemize}
\tightlist
\item
  We first removed observations with incomplete age and fleet data,
  retaining 375 observations.
\item
  Month information was used to assign quarters.
\item
  There was no discrimination between the set type for the purse seine
  fishery. So we assigned observations \(\leq\) 80 cm as \emph{LS} (log
  school) and \textgreater{} 80 cm as \emph{FS} free school.
\item
  Fork length was grouped into the length bins used in the assessment
  model.
\item
  Fish older than 7 years was grouped into a single group (age 7).
\item
  Region was assigned based on geographical location.
\item
  There were several observations ( \(\sim\) 37\%) with no geographical
  location, which were assigned to region 1b.
\end{itemize}

Figure~\ref{fig-caal} shows the CAAL information included in the
assessment model per fishery. Adults were mostly sampled from the
\emph{LL}, \emph{FS}, and \emph{HD} fisheries. Conversely, juveniles
were mainly present in the \emph{LS} fishery. The lengths ranged from 18
to 182 cm and all the model ages (i.e., quarters) were sampled. The
largest and oldest fish were observed in the \emph{GI}, \emph{LL}, and
\emph{FS} fisheries. We assumed no ageing error in the assessment model.

\subsection{Tagging}\label{tagging}

Tagging data was available for inclusion in the assessment model, which
consisted of yellowfin tuna tag releases and returns from the Indian
Ocean Tuna Tagging Programme (IOTTP) and from the main phase of the
Regional Tuna Tagging Project-Indian Ocean (RTTP-IO) conducted during
2005−2009. The IOTC has compiled all the release and recovery data from
the RTTP-IO and the complementary small-scale programmes in a single
database. A total of 54688 yellowfin tuna were released by the RTTP-IO
program. Most of the tag releases occurred within the western equatorial
region (region 1b) and a high proportion of these releases occurred in
the second and third quarters of 2006 (Figure XX). Limited tagging also
occurred within regions 1a and 2. The model included all tag recoveries
up to the end of 2014 and there were no further recoveries since the
last assessment. The spatial distributions of tag releases and
recoveries are presented in Figure XX.

A total of 9916 tag recoveries (after removing tags with unknown
recovery date or length) could be assigned to the fisheries included in
the assessment model. Almost all of the tags released in region 1 were
recovered in the home region, although some recoveries occurred in
adjacent regions, particularly in region 2. A small number of tags were
recovered in region 4 (from tags released in region 1b) and there were
no tags recovered from region 3 (Table XX). Most of the tag recoveries
occurred between mid-2006 and mid 2008 (Figure XX). The number of tag
recoveries started to attenuate in 2009 although small numbers of tags
were recovered up to the end of 2014.

Most of the tags were recovered by the purse seine fishery within region
1b (Figure XX). A significant proportion (35\%) of the tag returns from
purse seiners were not accompanied by information concerning the set
type. These tag recoveries were assigned to either the free-school or
log fishery based on the expected size of fish at the time of recapture;
i.e.~fish larger than 80 cm at release were assumed to be recaptured by
the free-school fishery; fish smaller than 80 cm at release and
recaptured within 18 months at liberty were assumed to be recovered by
the floating object fishery; fish smaller than 80 cm at release and
recaptured after 18 months at liberty were assumed to be recovered by
the free-school fishery.

For incorporation into the assessment model, tag releases were
stratified by release region, time period of release (quarter) and age
class. The recaptures by fishery for each release group inform the
assessment model on fishing mortality and abundance and fish movement.
Therefore, factors that might have affected the interpretation of tag
returns need to be accounted for to minimise potential bias. Fu
(\citeproc{ref-fuTagDataProcessing2020}{2020}) provides a summary of how
the tag data were incorporated into the assessments of IOTC tropical
tuna species, and below is a description of the procedure applied to
yellowfin tuna.

\subsubsection{Age assignment of tag
release}\label{age-assignment-of-tag-release}

The age at release was assumed based on the fish length at release and
the average length-at-age from the yellowfin growth function (see
Section XX). Fish aged 15 quarters and older were aggregated in a single
age group. Tag releases in regions 1a and 1b were stratified in separate
release groups due to the spatial separation of the individual release
events. A total of 54392 releases were classified into 131 tag release
groups. Most of the tag releases were in the 5−8 quarter age classes
(Figure XX).

\subsubsection{Initial tagging
mortality}\label{initial-tagging-mortality}

Hoyle et al. (\citeproc{ref-hoyleCovariatesReleaseMortality2015}{2015})
examined the effects of various covariates (e.g., individual tagger
effect) on tag failures for the RTTP program and estimated a combined
effect of 20\% for all tropical tuna species relative to a base failure
rate. No formal estimate was made for the base failure rate but a 7.5\%
was suggested by the WPTT in 2018 based on the assessment of the western
and central Pacific tuna species. This equates to a total tag failure
rate of 27.5\%. For the current assessment, the number of tags in each
release group was reduced by 27.5\% to account for initial tag
mortality.

\subsubsection{Chronic tag loss}\label{chronic-tag-loss}

Tag recoveries were also corrected for long-term tag loss (tag shedding)
based on an update of the analysis of Gaertner and Hallier
(\citeproc{ref-gaertnerTagSheddingTropical2015}{2015}). Tag loss for
yellowfin was estimated to be approximately 20\% at 2000 days at
liberty. This was accounted for through the SS3 chronical tag loss
parameter (an annual rate of 0.03).

\subsubsection{Reporting rate}\label{reporting-rate}

The returns from tag release group were classified by recapture fishery
and recapture time period (quarter). The results of associated tag
seeding experiments conducted during 2005−2008, have revealed
considerable temporal variability in tag reporting rates from the IO
purse-seine fishery
(\citeproc{ref-hillaryTagSheddingReporting2008}{Hillary et al. 2008}).
Reporting rates were lower in 2005 (57\%) compared to 2006 and 2007
(89\% and 94\%). Quarter estimates were also available but was similar
in magnitude (\citeproc{ref-hillaryReportingRateAnalyses2008}{Hillary,
IOTC, and Areso 2008}). This large increase over time was the result of
the development of publicity campaign and tag recovery scheme raising
the awareness of the stakeholders, i.e.~stevedores and crew. SS3 assumes
a constant fishery-specific reporting rate. To account for the temporal
change in reporting rate, the number of tag returns from the purse-seine
fishery in each stratum (tag group, year/quarter, and length class) were
corrected using the respective estimate of the reporting rate. Following
Kolody, Herrera, and Million
(\citeproc{ref-kolodyIndianOceanSkipjack2011}{2011}), Fu
(\citeproc{ref-fuTagDataProcessing2020}{2020}), and Fu
(\citeproc{ref-fuIndianOceanSkipjack2017}{2017}), an 100\% reporting
rate was assumed for at-sea recoveries whereas tags recovered from
Seychelles landings were corrected for reporting rates based on the
quarterly estimates from Hillary, IOTC, and Areso
(\citeproc{ref-hillaryReportingRateAnalyses2008}{2008}), and were also
corrected for the portion of the total purse-seine catches examined for
tags, based the proportions of EU purse seine catch landed in the
Seychelles relative to the total EU purse seine catches
(\citeproc{ref-kolodyIndianOceanSkipjack2011}{Kolody, Herrera, and
Million 2011}). For example, the adjusted number of observed recaptures
for a LS fishery as input to the model, the reporting rate (\(R'_L\))
was calculated using the following equation:

\[R'_L = R_L^{sea} + \frac{R_L^{sez}}{P^{sez}r^{sez}}\]

where:

\begin{itemize}
\tightlist
\item
  \(R_L^{sea}\) is the number of observed recaptures recovered at sea
  for the LS fishery
\item
  \(R_L^{sez}\) is the number of observed recaptures recovered in
  Seychelles for the LS fishery
\item
  \(r^{sez}\) is the reporting rates for purse seine tags removed from
  the Seychelles
\item
  \(P^{sez}\) is the scaling factor to account for the EU purse seine
  recaptures not landed in Seychelles
\end{itemize}

The adjusted number of recaptures for a FS fishery was calculated
similarly. The SS3 reporting parameters for the purse seine fisheries
were subsequently fixed at 100\% in the model. Some of the other (non
purse-seine) fisheries also returned a substantial number of tags. There
are no direct estimates of fishery specific reporting rates for these
fisheries. The reporting rates for these fisheries are estimated within
the assessment model.

\subsubsection{Small-scale tagging
programmes}\label{small-scale-tagging-programmes}

Additional tag release/recovery data are available from a number of
small-scale tagging programmes. The data set included a total of 7,828
tags released during 2002-08, primarily within regions 1b (70\%) and 4
(28\%). A total of 366 tag recoveries were reported, predominantly from
the Bait boat fishery in region 1a. There has been no comprehensive
analysis of these data and there is no information available concerning
the fishery specific reporting rate of these tags. The tag
release/recovery data from the SS tagging programmes were not
incorporated in the current range of assessment models. Earlier analysis
indicated that the stock assessment results were relatively insensitive
to the inclusion of these data
(\citeproc{ref-langleyStockAssessmentYellowfin2012}{Langley, Herrera,
and Million 2012}).

Fu et al. (\citeproc{ref-fuPreliminaryIndianOcean2018}{2018})
investigated a range of alternative options for processing and
incorporating the tagging data into the assessment model (see Table 5 of
Fu et al. (\citeproc{ref-fuPreliminaryIndianOcean2018}{2018})). These
exploratory analyses are not repeated in the current assessment.

\subsection{Environmental data}\label{environmental-data}

The 2018 assessment included a range of environmental data to
investigate the potential for the incorporation of environmental
covariates to inform the movement of fish. However, although there is
evidence that there may be an association between the movement of
yellowfin tuna and seasonal and temporal changes in ocean conditions in
the IO, the potential relationship between environmental indices and
fish movement is unclear. Langley
(\citeproc{ref-langleyUpdate2015Indian2016}{2016}) and Fu et al.
(\citeproc{ref-fuPreliminaryIndianOcean2018}{2018}) suggested that these
environmental indices had no influence on the estimation of yellowfin
tuna movement rates of different life stages between adjacent model
regions, and seasonal variation in movement may be better accounted for
by models that can explicitly incorporate seasonal effects
(\citeproc{ref-fuPreliminaryIndianOcean2018}{Fu et al. 2018}).
Therefore, environmental information was not included in the 2021
assessment.

A significant negative association between the Indian Ocean Dipoles
(IODs) and the catch rates of yellowfin tuna was observed
(\citeproc{ref-lanEffectsClimateVariability2013}{Lan, Evans, and Lee
2013}; \citeproc{ref-lanInfluenceOceanographicClimatic2020}{Lan, Chang,
and Wu 2020}), as well as an impact of the El Niño Southern Oscillation
(ENSO) on catch rates near the Arabian Sea
(\citeproc{ref-lanInfluenceOceanographicClimatic2020}{Lan, Chang, and Wu
2020}). Langley, Fu, and Maunder
(\citeproc{ref-langleyInvestigationRecruitmentDynamics2023}{2023}) found
that periods of strong recruitment in regions 1 and 4 appear to
correspond with oceanographic conditions indexed by Dipole Mode Index
(DMI), but they did not find apparent correspondence between yearly
trends in IO environmental conditions and the LL CPUE indices from each
of the model regions. There is no strong indication that the
catchability of the equatorial longline fisheries (region 1 and 4) is
strongly influenced by the prevailing environmental conditions, but
there is some indication that oceanographic conditions may influence
short-term (1-2 yr) variation in longline catchability in the sub
tropical regions (region 2 and 3)
(\citeproc{ref-langleyInvestigationRecruitmentDynamics2023}{Langley, Fu,
and Maunder 2023}). Variability in deviates of movement rates between
region 1 and 4 is broadly consistent with the fluctuations in the DMI
with higher movement estimated under positive IOD conditions.

Since no strong relationships between IO environmental conditions and
the yellowfin dynamics have been identified, we did not incorporate any
environmental variable in the current assessment; however, we suggest
further studies on this topic.

\section{Model parameters}\label{model-parameters}

\subsection{Population dynamics}\label{population-dynamics}

Parametrization of recruitment, growth, etc in SS.

\subsection{Fishery dynamics}\label{fishery-dynamics}

Selectivity principally.

\subsection{Tagged fish}\label{tagged-fish}

Tag reporting, mixing, etc.

\subsection{Likelihood components}\label{likelihood-components}

Weights, error structure, observation error, etc.

\subsection{Parameter estimation and
uncertainty}\label{parameter-estimation-and-uncertainty}

Hessian, max gradient, jitter analysis, likelihood profiles, etc.

\subsection{Stock status}\label{stock-status}

Related to MSY, depletion trends, Kobe, etc.

\section{Model runs}\label{model-runs}

\subsection{Update from the last
assessment}\label{update-from-the-last-assessment}

Describe stepwise model developtment, etc.

\subsection{Sensitivity and structural
uncertainty}\label{sensitivity-and-structural-uncertainty}

Different steepness, M, etc, and spatial configurations.

\section{Model results}\label{model-results}

\subsection{Fits}\label{fits}

Describe fits to data.

\subsection{Parameter estimates}\label{parameter-estimates}

Describe parameter estimates.

\subsection{Time series}\label{time-series}

Recruitment, SSB, etc.

\subsection{Sensitivity}\label{sensitivity}

Describe sensitivity analyses.

\section{Discussion and conclusions}\label{discussion-and-conclusions}

Discuss. Recommendations for future assessments.

\section{Acknowledgements}\label{acknowledgements}

Write your acknowledgements.

\newpage{}

\section{Tables}\label{tables}

\begin{longtable}[]{@{}ll@{}}
\caption{Grid size categories in the \emph{original} size
dataset.}\label{tbl-grid-size}\tabularnewline
\toprule\noalign{}
Grid category & Resolution (latitude \(\times\) longitude) \\
\midrule\noalign{}
\endfirsthead
\toprule\noalign{}
Grid category & Resolution (latitude \(\times\) longitude) \\
\midrule\noalign{}
\endhead
\bottomrule\noalign{}
\endlastfoot
9 & \(30^\circ\times 30^\circ\) \\
A & \(10^\circ\times 20^\circ\) \\
7 & \(10^\circ\times 10^\circ\) \\
8 & \(20^\circ\times 20^\circ\) \\
5 & \(1^\circ\times 1^\circ\) \\
6 & \(5^\circ\times 5^\circ\) \\
NJA\_SYC & Seychelles National Jurisdiction Area \\
\end{longtable}

\newpage{}

\begin{longtable}[]{@{}ll@{}}
\caption{Nine fishery groups and codes used in the current
assessment.}\label{tbl-fishery-codes}\tabularnewline
\toprule\noalign{}
Fishery code & Fishery group \\
\midrule\noalign{}
\endfirsthead
\toprule\noalign{}
Fishery code & Fishery group \\
\midrule\noalign{}
\endhead
\bottomrule\noalign{}
\endlastfoot
GI & Gillnet \\
HD & Handline \\
LL & Longline \\
OT & Others \\
BB & Baitboat \\
FS & Purse seine, free school \\
LS & Purse seine, log school \\
TR & Troll \\
LF & Longline (fresh tuna) \\
\end{longtable}

\newpage{}

\begin{longtable}[]{@{}ll@{}}

\caption{\label{tbl-fleet-4A}Fishery definition in the four-areas
assessment configuration.}

\tabularnewline

\toprule\noalign{}
Fishery number & Fishery code and region \\
\midrule\noalign{}
\endhead
\bottomrule\noalign{}
\endlastfoot
1 & GI 1a \\
2 & HD 1a \\
3 & LL 1a \\
4 & OT 1a \\
5 & BB 1b \\
6 & FS 1b \\
7 & LL 1b \\
8 & LS 1b \\
9 & TR 1b \\
10 & LL 2 \\
11 & LL 3 \\
12 & GI 4 \\
13 & LL 4 \\
14 & OT 4 \\
15 & TR 4 \\
16 & FS 2 \\
17 & LS 2 \\
18 & TR 2 \\
19 & FS 4 \\
20 & LS 4 \\
21 & LF 4 \\

\end{longtable}

\newpage{}

\section{Figures}\label{figures}

\begin{figure}

\centering{

\includegraphics[width=6.69in,height=\textheight]{C:/Use/OneDrive - AZTI/General - IOTC_YFT_2024/output/figures/map_IO_4ARegions.png}

}

\caption{\label{fig-4A-config}Model areas or regions used in the
four-areas assessment configuration. The region 1 is divided into two
sub-regions implicitly modelled in the assessment model using the
fleets-as-areas approach.}

\end{figure}%

\newpage{}

\begin{figure}

\centering{

\includegraphics[width=6.69in,height=\textheight]{C:/Use/OneDrive - AZTI/General - IOTC_YFT_2024/output/figures/ts_catch.png}

}

\caption{\label{fig-catch-bar}Total annual catch of yellowfin tuna by
fishery group from 1950 to 2023. Fishery group codes are described in
Table~\ref{tbl-fishery-codes}.}

\end{figure}%

\newpage{}

\begin{figure}

\centering{

\includegraphics[width=6.69in,height=\textheight]{C:/Use/OneDrive - AZTI/General - IOTC_YFT_2024/output/figures/agg_size.png}

}

\caption{\label{fig-agg-size}Size compositions per fishery included in
the assessment model. Colored lines are size compositions obtained using
simple aggregation while black lines used the catch-raised aggregation.
Size compositions were summed over time.}

\end{figure}%

\newpage{}

\begin{figure}

\centering{

\includegraphics[width=6.69in,height=\textheight]{C:/Use/OneDrive - AZTI/General - IOTC_YFT_2024/output/figures/mlen.png}

}

\caption{\label{fig-mlen}Mean length per quarter per fishery. Colored
lines used the simple aggregation while black lines used the
catch-raised aggregation.}

\end{figure}%

\newpage{}

\begin{figure}

\centering{

\includegraphics[width=6.69in,height=\textheight]{C:/Use/OneDrive - AZTI/General - IOTC_YFT_2024/output/figures/catch_grid.png}

}

\caption{\label{fig-catch-grid}Spatial distribution of IO catches per
fishery group. The pie radius represents the aggregated catch from 1980
to 2023. Fishery group codes are described in
Table~\ref{tbl-fishery-codes}.}

\end{figure}%

\newpage{}

\begin{figure}

\centering{

\includegraphics[width=6.69in,height=\textheight]{C:/Use/OneDrive - AZTI/General - IOTC_YFT_2024/output/figures/size_grid.png}

}

\caption{\label{fig-size-grid}Spatial distribution of size compositions
per fishery group. The size compositions were aggregated over time.
Fishery group codes are described in Table~\ref{tbl-fishery-codes}.}

\end{figure}%

\newpage{}

\begin{figure}

\centering{

\includegraphics[width=6.69in,height=\textheight]{C:/Use/OneDrive - AZTI/General - IOTC_YFT_2024/output/figures/rq_size.png}

}

\caption{\label{fig-rq-size}The availability of size composition data
from each fishery by quarter. The size of the bubble indicates the
reporting quality score (small size = high quality, large size = low
quality).}

\end{figure}%

\newpage{}

\begin{figure}

\centering{

\includegraphics[width=6.69in,height=\textheight]{C:/Use/OneDrive - AZTI/General - IOTC_YFT_2024/output/figures/map_size_grid_4A.png}

}

\caption{\label{fig-map-grid-agg1}Example of grid categories in the
original size dataset. The region assignment was done based on the grid
centroid.}

\end{figure}%

\newpage{}

\begin{figure}

\centering{

\includegraphics[width=6.69in,height=\textheight]{C:/Use/OneDrive - AZTI/General - IOTC_YFT_2024/output/figures/map_size_grid_std_4A.png}

}

\caption{\label{fig-map-grid-agg2}Example of grids in the cwp55 size
dataset. The region assignment was done based on the grid centroid.}

\end{figure}%

\newpage{}

\begin{figure}

\centering{

\includegraphics[width=3.34in,height=\textheight]{C:/Use/OneDrive - AZTI/General - IOTC_YFT_2024/output/figures/imputation_grid_5.png}

}

\caption{\label{fig-imputation}Percentage of size observations by
imputation level. Level 0 means perfect match, so no imputation was
required.}

\end{figure}%

\newpage{}

\begin{figure}

\centering{

\includegraphics[width=6.69in,height=\textheight]{C:/Use/OneDrive - AZTI/General - IOTC_YFT_2024/output/figures/compare_catch.png}

}

\caption{\label{fig-comp-catch}Comparison between the catch values used
in the 2021 assessment and the current (2024) values.}

\end{figure}%

\newpage{}

\begin{figure}

\centering{

\includegraphics[width=6.69in,height=\textheight]{C:/Use/OneDrive - AZTI/General - IOTC_YFT_2024/output/figures/ts_cpue_area.png}

}

\caption{\label{fig-ts-cpue}Scaled CPUE LL time serie per region. The
shaded area is the 95\% confidence interval.}

\end{figure}%

\newpage{}

\begin{figure}

\centering{

\includegraphics[width=3.34in,height=\textheight]{C:/Use/OneDrive - AZTI/General - IOTC_YFT_2024/output/figures/ts_ps_cpue.png}

}

\caption{\label{fig-ts-ps-cpue}Scaled purse seine CPUE time series. Both
series were used for region 1b. The shaded area is the 95\% confidence
interval.}

\end{figure}%

\newpage{}

\begin{figure}

\centering{

\includegraphics[width=6.69in,height=\textheight]{C:/Use/OneDrive - AZTI/General - IOTC_YFT_2024/output/figures/compare_cpue.png}

}

\caption{\label{fig-comp-cpue}Comparison between the scaled CPUE LL
values used in the 2021 assessment and the current (2024) values.}

\end{figure}%

\newpage{}

\begin{figure}

\centering{

\includegraphics[width=6.69in,height=\textheight]{C:/Use/OneDrive - AZTI/General - IOTC_YFT_2024/output/figures/compare_size.png}

}

\caption{\label{fig-comp-agg-size}Comparison between the aggregated size
compositions per fishery used in the 2021 and the current (2024)
assessment.}

\end{figure}%

\newpage{}

\begin{figure}

\centering{

\includegraphics[width=6.69in,height=\textheight]{C:/Use/OneDrive - AZTI/General - IOTC_YFT_2024/output/figures/compare_mlen.png}

}

\caption{\label{fig-comp-mlen}Comparison between the mean length per
quarter per fishery used in the 2021 and in the current (2024)
assessment.}

\end{figure}%

\newpage{}

\begin{figure}

\centering{

\includegraphics[width=3.34in,height=\textheight]{C:/Use/OneDrive - AZTI/General - IOTC_YFT_2024/output/figures/caal_nsamp.png}

}

\caption{\label{fig-caal-nsamp}Number of sampled fish per year and
fishery group in the age-length dataset from the GERUNDIO project.}

\end{figure}%

\newpage{}

\begin{figure}

\centering{

\includegraphics[width=6.69in,height=\textheight]{C:/Use/OneDrive - AZTI/General - IOTC_YFT_2024/output/figures/caal.png}

}

\caption{\label{fig-caal}Conditional age-at-length (CAAL) data included
in the assessment model. The bubble size represents the proportion of
ages at a given length.}

\end{figure}%

\newpage{}

\section{Appendix}\label{appendix}

\subsection{Acronyms and
Abbreviations}\label{acronyms-and-abbreviations}

Include a table with acronyms (e.g., GLMM, SS, etc.)

\newpage{}

\subsection{Clustering of size
compositions}\label{clustering-of-size-compositions}

Add text here.

\newpage{}

\subsection{One-area model
configuration}\label{one-area-model-configuration}

Add text here.

\newpage{}

\subsection{Two-areas model
configuration}\label{two-areas-model-configuration}

Add text here.

\newpage{}

\section*{References}\label{references}
\addcontentsline{toc}{section}{References}

\phantomsection\label{refs}
\begin{CSLReferences}{1}{0}
\bibitem[\citeproctext]{ref-artetxe-arrateOtolithStableIsotopesinreview}
Artetxe-Arrate, Iraide, Igaratza Fraile, Patricia Lastra-Luque, Jessica
Farley, Naomi Clear, Umair Shahid, S. Adbul-Razzaue, et al. in review.
{``Otolith Stable Isotopes Highlight the Importance of Local Nursery
Areas as the Origin of Recruits to Yellowfin Tuna ({Thunnus} Albacares)
Fisheries in the Western {Indian Ocean}.''} \emph{Fisheries Research},
in review.

\bibitem[\citeproctext]{ref-artetxe-arrateReviewFisheriesLife2021}
Artetxe-Arrate, Iraide, Igaratza Fraile, Francis Marsac, Jessica H.
Farley, Naiara Rodriguez-Ezpeleta, Campbell R. Davies, Naomi P. Clear,
Peter Grewe, and Hilario Murua. 2021. {``A Review of the Fisheries, Life
History and Stock Structure of Tropical Tuna (Skipjack {Katsuwonus}
Pelamis, Yellowfin {Thunnus} Albacares and Bigeye {Thunnus} Obesus) in
the {Indian Ocean}.''} In \emph{Advances in {Marine Biology}},
88:39--89. Elsevier. \url{https://doi.org/10.1016/bs.amb.2020.09.002}.

\bibitem[\citeproctext]{ref-chassotAreThereSmall2014}
Chassot, Emmanuel. 2014. {``Are There Small Yellowfin Caught by Purse
Seiners in Free-Swimming Schools?''} IOTC-2014-WPDCS10-INF05. Indian
Ocean Tuna Comission.

\bibitem[\citeproctext]{ref-correaStandardizedCatchUnit2024}
Correa, Giancarlo M., Jon Uranga, David Kaplan, Gorka Merino, and
Lourdes Ramos. 2024. {``Standardized Catch Per Unit Effort of Yellowfin
Tuna in the {Indian Ocean} for the {European} Purse Seine Fleet
Operating on Floating Objects.''} IOTC-2024-WPTT26(DP)-11rev1. Indian
Ocean Tuna Comission.

\bibitem[\citeproctext]{ref-dammannagodaEvidenceFineGeographical2008}
Dammannagoda, Sudath T., David A. Hurwood, and Peter B. Mather. 2008.
{``Evidence for Fine Geographical Scale Heterogeneity in Gene
Frequencies in Yellowfin Tuna ({Thunnus} Albacares) from the North
{Indian Ocean} Around {Sri Lanka}.''} \emph{Fisheries Research} 90
(1-3): 147--57. \url{https://doi.org/10.1016/j.fishres.2007.10.006}.

\bibitem[\citeproctext]{ref-duffyGlobalTrophicEcology2017}
Duffy, Leanne M., Petra M. Kuhnert, Heidi R. Pethybridge, Jock W. Young,
Robert J. Olson, John M. Logan, Nicolas Goñi, et al. 2017. {``Global
Trophic Ecology of Yellowfin, Bigeye, and Albacore Tunas:
{Understanding} Predation on Micronekton Communities at Ocean-Basin
Scales.''} \emph{Deep Sea Research Part II: Topical Studies in
Oceanography} 140 (June): 55--73.
\url{https://doi.org/10.1016/j.dsr2.2017.03.003}.

\bibitem[\citeproctext]{ref-farleyUpdatingEstimationAge2023}
Farley, Jessica H., Kyne KrusicGolub, Paige Eveson, Patricia Luque,
Igaratza Fraile, Iraide Artetxe-Arrate, Iker Zudaire, et al. 2023.
{``Updating the Estimation of Age and Growth of Yellowfin Tuna
({Thunnus} Albacares) in the {Indian Ocean} Using Otoliths.''}
IOTC-2023-WPTT25-20. Indian Ocean Tuna Comission.

\bibitem[\citeproctext]{ref-farleyEstimatingAgeGrowth2021}
Farley, Jessica, Kyne Krusic-Golub, Paige Eveson, Patricia Luque, Naomi
Clear, Igaratza Fraile, Iraide Artetxe-Arrate, et al. 2021.
{``Estimating the Age and Growth of Yellowfin Tuna ({Thunnus} Albacares)
in the {Indian Ocean} from Counts of Daily and Annual Increments in
Otoliths.''} IOTC-2021-WPTT23-05\_Rev1. Indian Ocean Tuna Comission.

\bibitem[\citeproctext]{ref-farleyAgeGrowthYellowfin2020}
Farley, Jessica, Krusic-Golub, Paige Eveson, Naomi Clear, F. Rouspard,
C. Sanchez, Simon Nicol, and John Hampton. 2020. {``Age and Growth of
Yellowfin and Bigeye Tuna in the Western and Central {Pacific Ocean}
from Otoliths.''} WCPFC-SC16-2020/SC16-SA-WP-02. {Western and Central
Pacific Fisheries Commission}.

\bibitem[\citeproctext]{ref-froeseFishBase2024}
Froese, Rainer, and Daniel Pauly. 2024. {``{FishBase}.''} World \{\{Wide
Web\}\} Electronic Publication. www.fishbase.org.

\bibitem[\citeproctext]{ref-fuIndianOceanSkipjack2017}
Fu, Dan. 2017. {``Indian Ocean Skipjack Tuna Stock Assessment 1950-2016
({Stock Synthesis}).''} IOTC-2017-WPTT19-47. Indian Ocean Tuna
Comission.

\bibitem[\citeproctext]{ref-fuTagDataProcessing2020}
---------. 2020. {``Tag Data Processing for {IOTC} Tropical Tuna
Assessments.''} IOTC-2020-WPTT22(DP)-10. Indian Ocean Tuna Comission.

\bibitem[\citeproctext]{ref-fuPreliminaryIndianOcean2018}
Fu, Dan, Adam D. Langley, Gorka Merino, and Agurtzane Urtizberea. 2018.
{``Preliminary {Indian Ocean Yellowfin Tuna Stock Assessment} 1950-2017
({Stock Synthesis}).''} IOTC--2018--WPTT20--33. Indian Ocean Tuna
Comission.

\bibitem[\citeproctext]{ref-fuPreliminaryIndianYellowfin2021}
Fu, Dan, Agurtzane Urtizberea Ijurco, Massimiliano Cardinale, Richard D
Methot, Simon D. Hoyle, and Gorka Merino. 2021. {``Preliminary {Indian
Yellowfin} Tuna Stock Assessment 1950-2020 ({Stock Synthesis}).''}
IOTC-2021-WPTT23-12. Indian Ocean Tuna Comission.

\bibitem[\citeproctext]{ref-gaertnerTagSheddingTropical2015}
Gaertner, Daniel, and Jean Pierre Hallier. 2015. {``Tag Shedding by
Tropical Tunas in the {Indian Ocean} and Other Factors Affecting the
Shedding Rate.''} \emph{Fisheries Research} 163 (March): 98--105.
\url{https://doi.org/10.1016/j.fishres.2014.02.025}.

\bibitem[\citeproctext]{ref-greehanReviewLengthFrequency2013}
Greehan, James, and Simon D. Hoyle. 2013. {``Review of Length Frequency
Data of the {Taiwanese Distant Water Longline Fleet}.''}
IOTC-2013-WPDCS09-12. Indian Ocean Tuna Comission.

\bibitem[\citeproctext]{ref-greweGeneticPopulationConnectivity2020}
Grewe, Peter, Pierre Feutry, Scott Foster, Aulich, Matt Lansdell, Scott
Cooper, Naomi Clear, et al. 2020. {``Genetic Population Connectivity of
Yellowfin Tuna in the {Indian Ocean} from the {PSTBS-IO Project}.''}
IOTC-2020-WPTT22(AS)12\_REV1. Seychelles: Indian Ocean Tuna Comission.

\bibitem[\citeproctext]{ref-herreraProposalSystemAssess2010}
Herrera, Miguel. 2010. {``Proposal for a System to Assess the Quality of
Fisheries Statistics at the {IOTC}.''} IOTC-2010-WPDCS-06. Indian Ocean
Tuna Comission.

\bibitem[\citeproctext]{ref-hillaryReportingRateAnalyses2008}
Hillary, R., Secretariat IOTC, and J. Areso. 2008. {``Reporting Rate
Analyses for Recaptures from {Seychelles} Port for Yellowfin, Bigeye and
Skipjack Tuna.''} IOTC-2008-WPTT-18. Indian Ocean Tuna Comission.

\bibitem[\citeproctext]{ref-hillaryTagSheddingReporting2008}
Hillary, R., J. Million, A. Anganuzzi, and J. Areso. 2008. {``Tag
Shedding and Reporting Rate Estimates for {Indian Ocean} Tuna Using
Double-Tagging and Tag-Seeding Experiments.''} IOTC-2008-WPTDA-04.
Indian Ocean Tuna Comission.

\bibitem[\citeproctext]{ref-hosseiniInvestigationsReproductiveBiology2016}
Hosseini, S A, and F Kaymaram. 2016. {``Investigations on the
Reproductive Biology and Diet of Yellowfin Tuna, {Thunnus} Albacares,
({Bonnaterre}, 1788) in the {Oman Sea}.''} \emph{Journal of Applied
Icthyology} 32: 310--17. \url{https://doi.org/10.1111/jai.12907}.

\bibitem[\citeproctext]{ref-hoyleIndianOceanTropical2018}
Hoyle, Simon D. 2018. {``Indian {Ocean} Tropical Tuna Regional Scaling
Factors That Allow for Seasonality and Cell Areas.''}
IOTC-2018-WPM09-13. Indian Ocean Tuna Comission.

\bibitem[\citeproctext]{ref-hoyleReviewSizeData2021}
---------. 2021. {``Review of Size Data from {Indian Ocean} Longline
Fleets, and Its Utility for Stock Assessment.''} IOTC-2021-WPTT23-07.
Indian Ocean Tuna Comission.

\bibitem[\citeproctext]{ref-hoyleEffortCreepLongline2024}
---------. 2024. {``Effort Creep in Longline and Purse Seine {CPUE} and
Its Application in Tropical Tuna Assessments.''}
IOTC-2024-WPTT26(DP)-16. Indian Ocean Tuna Comission.

\bibitem[\citeproctext]{ref-hoyleCatchUnitEffort2024}
Hoyle, Simon D., Robert A. Campbell, Nicholas D. Ducharme-Barth, Arnaud
Grüss, Bradley R. Moore, James T. Thorson, Laura Tremblay-Boyer, Henning
Winker, Shijie Zhou, and Mark N. Maunder. 2024. {``Catch Per Unit Effort
Modelling for Stock Assessment: {A} Summary of Good Practices.''}
\emph{Fisheries Research} 269 (January): 106860.
\url{https://doi.org/10.1016/j.fishres.2023.106860}.

\bibitem[\citeproctext]{ref-hoyleCovariatesReleaseMortality2015}
Hoyle, Simon D., Bruno M. Leroy, Simon J. Nicol, and W. John Hampton.
2015. {``Covariates of Release Mortality and Tag Loss in Large-Scale
Tuna Tagging Experiments.''} \emph{Fisheries Research} 163 (March):
106--18. \url{https://doi.org/10.1016/j.fishres.2014.02.023}.

\bibitem[\citeproctext]{ref-hoyleExploringPossibleCauses2017}
Hoyle, Simon D., Keisuke Satoh, and Takayuki Matsumoto. 2017.
{``Exploring Possible Causes of Historical Discontinuities in {Japanese}
Longline {CPUE}.''} IOTC-2017-WPTT19-33. Indian Ocean Tuna Comission.

\bibitem[\citeproctext]{ref-iotcReviewYellowfinTuna2021}
IOTC, Secretariat. 2021. {``Review of {Yellowfin Tuna Statistical
Data}.''} IOTC-2021-WPTT23(DP)-07\_Rev1. Indian Ocean Tuna Comission.

\bibitem[\citeproctext]{ref-iotcReviewStatisticalData2024}
---------. 2024. {``Review of the Statistical Data Available for
Yellowfin Tuna (1950-2022).''} IOTC-2024-WPTT26(DP)-07. Indian Ocean
Tuna Comission.

\bibitem[\citeproctext]{ref-kaplanStandardizedCPUEAbundance2024}
Kaplan, David, Giancarlo M Correa, Lourdes Ramos, Antoine Duparc, Jon
Uranga, Josu Santiago, Laurent Floch, et al. 2024. {``Standardized
{CPUE} Abundance Indices for Adult Yellowfin Tuna Caught in
Free-Swimming School Sets by the {European} Purse-Seine Fleet in the
{Indian Ocean}, 1991-2022.''} IOTC-2024-WPTT26(DP)-13rev2.

\bibitem[\citeproctext]{ref-kolodyEstimationIndianOcean2018}
Kolody, D. 2018. {``Estimation of {Indian Ocean Skipjack Purse Seine
Catchability Trends} from {Bigeye} and {Yellowfin Assessments}.''}
IOTC--2018--WPTT20--32. Indian Ocean Tuna Comission.

\bibitem[\citeproctext]{ref-kolodyIndianOceanSkipjack2011}
Kolody, D., Miguel Herrera, and Julien Million. 2011. {``Indian {Ocean
Skipjack Tuna Stock Assessment} 1950-2009 ({Stock Synthesis}).''}
IOTC-2011-WPTT13-31. Indian Ocean Tuna Comission.

\bibitem[\citeproctext]{ref-krishnanDietCompositionFeeding2024}
Krishnan, Silambarasan, Tiburtius Antony Pillai, John Chembian Antony
Rayappan, Tharumar Yagappan, and Jeyabaskaran Rajapandian. 2024. {``Diet
Composition and Feeding Habits of Yellowfin Tuna {\emph{Thunnus}}{
\emph{Albacares}} ({Bonnaterre}, 1788) from the {Bay} of {Bengal}.''}
\emph{Aquatic Living Resources} 37: 10.
\url{https://doi.org/10.1051/alr/2024008}.

\bibitem[\citeproctext]{ref-kumarReproductiveDynamicsYellowfin2022}
Kumar, Mamidi Satish, and Shubhadeep Ghosh. 2022. {``Reproductive
{Dynamics} of {Yellowfin Tuna}, {Thunnus} Albacares ({Bonnaterre} 1788)
{Exploited} from {Western Bay} of {Bengal}.''} \emph{Thalassas: An
International Journal of Marine Sciences} 38: 1003--12.
\url{https://doi.org/10.1007/s41208-022-00429-1}.

\bibitem[\citeproctext]{ref-kunalMitochondrialDNAAnalysis2013}
Kunal, Swaraj Priyaranjan, Girish Kumar, Maria Rosalia Menezes, and Ram
Murti Meena. 2013. {``Mitochondrial {DNA} Analysis Reveals Three Stocks
of Yellowfin Tuna {Thunnus} Albacares ({Bonnaterre}, 1788) in {Indian}
Waters.''} \emph{Conservation Genetics} 14 (1): 205--13.
\url{https://doi.org/10.1007/s10592-013-0445-3}.

\bibitem[\citeproctext]{ref-lanInfluenceOceanographicClimatic2020}
Lan, Kuo-Wei, Yi-Jay Chang, and Yan-Lun Wu. 2020. {``Influence of
Oceanographic and Climatic Variability on the Catch Rate of Yellowfin
Tuna ({Thunnus} Albacares) Cohorts in the {Indian Ocean}.''} \emph{Deep
Sea Research Part II: Topical Studies in Oceanography} 175 (May):
104681. \url{https://doi.org/10.1016/j.dsr2.2019.104681}.

\bibitem[\citeproctext]{ref-lanEffectsClimateVariability2013}
Lan, Kuo-Wei, Karen Evans, and Ming-An Lee. 2013. {``Effects of Climate
Variability on the Distribution and Fishing Conditions of Yellowfin Tuna
({Thunnus} Albacares) in the Western {Indian Ocean}.''} \emph{Climatic
Change} 119 (1): 63--77.
\url{https://doi.org/10.1007/s10584-012-0637-8}.

\bibitem[\citeproctext]{ref-langleyStockAssessmentYellowfin2015}
Langley, Adam D. 2015. {``Stock Assessment of Yellowfin Tuna in the
{Indian Ocean} Using {Stock Synthesis}.''} IOTC--2015--WPTT17--30.
Indian Ocean Tuna Comission.

\bibitem[\citeproctext]{ref-langleyUpdate2015Indian2016}
---------. 2016. {``An Update of the 2015 {Indian Ocean Yellowfin Tuna}
Stock Assessment for 2016.''} IOTC-2016-WPTT18-27. Indian Ocean Tuna
Comission.

\bibitem[\citeproctext]{ref-langleyInvestigationRecruitmentDynamics2023}
Langley, Adam D., Dan Fu, and Mark Maunder. 2023. {``An Investigation of
the Recruitment Dynamics of {Indian Ocean} Yellowfin Tuna.''}
IOTC-2023-WPTT25-12. Indian Ocean Tuna Comission.

\bibitem[\citeproctext]{ref-langleyPreliminaryStockAssessment2008}
Langley, Adam D., John Hampton, Miguel Herrera, and Julien Million.
2008. {``Preliminary Stock Assessment of Yellowfin Tuna in the {Indian
Ocean} Using {MULTIFAN-CL}.''} IOTC-2008-WPTT-10. Indian Ocean Tuna
Comission.

\bibitem[\citeproctext]{ref-langleyStockAssessmentYellowfin2009}
Langley, Adam D., Miguel Herrera, Jean-Pierre Hallier, and Julien
Million. 2009. {``Stock Assessment of Yellowfin Tuna in the {Indian
Ocean} Using {MULTIFAN-CL}.''} IOTC-2009-WPTT-10. Indian Ocean Tuna
Comission.

\bibitem[\citeproctext]{ref-langleyStockAssessmentYellowfin2010}
Langley, Adam D., Miguel Herrera, and Julien Million. 2010. {``Stock
Assessment of Yellowfin Tuna in the {Indian Ocean} Using
{MULTIFAN-CL}.''} IOTC-2010-WPTT-23. Indian Ocean Tuna Comission.

\bibitem[\citeproctext]{ref-langleyStockAssessmentYellowfin2011}
---------. 2011. {``Stock Assessment of Yellowfin Tuna in the {Indian
Ocean} Using {MULTIFAN-CL}.''} IOTC-2011-WPTT-13. Indian Ocean Tuna
Comission.

\bibitem[\citeproctext]{ref-langleyStockAssessmentYellowfin2012}
---------. 2012. {``Stock Assessment of Yellowfin Tuna in the {Indian
Ocean} Using {MULTIFAN-CL}.''} IOTC-2012-WPTT-14-38 Rev\_1. Indian Ocean
Tuna Comission.

\bibitem[\citeproctext]{ref-leeUseConditionalAge2019}
Lee, Huihua, Kevin R. Piner, Ian G. Taylor, and Toshihide Kitakado.
2019. {``On the Use of Conditional Age at Length Data as a Likelihood
Component in Integrated Population Dynamics Models.''} \emph{Fisheries
Research} 216 (August): 204--11.
\url{https://doi.org/10.1016/j.fishres.2019.04.007}.

\bibitem[\citeproctext]{ref-matsumotoJointLonglineCPUE2024}
Matsumoto, Takayuki, Keisuke Satoh, Wen-Pei Tsai, Sheng-Ping Wang,
Jung-Hyun Lim, Heewon Park, and Sung Il Lee. 2024. {``Joint Longline
{CPUE} for Yellowfin Tuna in the {Indian Ocean} by the {Japanese},
{Korean} and {Taiwanese} Longline Fishery.''} IOTC-2024-WPTT26(DP)-14.
Indian Ocean Tuna Comission.

\bibitem[\citeproctext]{ref-maunderIndependentReviewRecent2023}
Maunder, Mark, Carolina V. Minte-Vera, Adam D. Langley, and Daniel
Howell. 2023. {``Independent Review of Recent {IOTC} Yellowfin Tuna
Assessment.''} IOTC-2023-WPTT25-13\_Rev1. Indian Ocean Tuna Comission.

\bibitem[\citeproctext]{ref-mckechnieStockAssessmentBigeye2017}
McKechnie, S., G. Pilling, and John Hampton. 2017. {``Stock Assessment
of Bigeye Tuna in the Western and Central {Pacific Ocean}.''}
WCPFC-SC13-2017/SA-WP-05. {Western and Central Pacific Fisheries
Commission}.

\bibitem[\citeproctext]{ref-menardIsotopicEvidenceDistinct2007}
Ménard, Frédéric, Anne Lorrain, Michel Potier, and Francis Marsac. 2007.
{``Isotopic Evidence of Distinct Feeding Ecologies and Movement Patterns
in Two Migratory Predators (Yellowfin Tuna and Swordfish) of the Western
{Indian Ocean}.''} \emph{Marine Biology} 153 (2): 141--52.
\url{https://doi.org/10.1007/s00227-007-0789-7}.

\bibitem[\citeproctext]{ref-methotRecommendationsConfigurationIndian2019}
Methot, Richard D. 2019. {``Recommendations on the Configuration of the
{Indian Ocean} Yellowfin Tuna Stock Assessment Model.''}

\bibitem[\citeproctext]{ref-methotStockSynthesisBiological2013}
Methot, Richard D., and Chantell R. Wetzel. 2013. {``Stock Synthesis:
{A} Biological and Statistical Framework for Fish Stock Assessment and
Fishery Management.''} \emph{Fisheries Research} 142 (May): 86--99.
\url{https://doi.org/10.1016/j.fishres.2012.10.012}.

\bibitem[\citeproctext]{ref-mooreMovementJuvenileTuna2019}
Moore, Bradley R, Pratiwi Lestari, Scott C Cutmore, Craig Proctor, and
Robert J G Lester. 2019. {``Movement of Juvenile Tuna Deduced from
Parasite Data.''} Edited by James Watson. \emph{ICES Journal of Marine
Science} 76 (6): 1678--89. \url{https://doi.org/10.1093/icesjms/fsz022}.

\bibitem[\citeproctext]{ref-muhlingReproductionLarvalBiology2017}
Muhling, Barbara A., John T. Lamkin, Francisco Alemany, Alberto García,
Jessica Farley, G. Walter Ingram, Diego Alvarez Berastegui, Patricia
Reglero, and Raul Laiz Carrion. 2017. {``Reproduction and Larval Biology
in Tunas, and the Importance of Restricted Area Spawning Grounds.''}
\emph{Reviews in Fish Biology and Fisheries} 27 (4): 697--732.
\url{https://doi.org/10.1007/s11160-017-9471-4}.

\bibitem[\citeproctext]{ref-nishidaStockAssessmentYellowfin2005}
Nishida, Tom, and Hiroshi Shono. 2005. {``Stock Assessment of Yellowfin
Tuna ({Thunnus} Albacares) Resources in the {Indian Ocean} by the Age
Structured Production Model ({ASPM}) Analyses.''} IOTC-2005-WPTT-09.
Indian Ocean Tuna Comission.

\bibitem[\citeproctext]{ref-nishidaStockAssessmentYellowfin2007}
---------. 2007. {``Stock Assessment of Yellowfin Tuna ({Thunnus}
Albacares) in the {Indian Ocean} by the Age Structured Production
Model({ASPM}) Analyses.''} IOTC-2007-WPTT-12. Indian Ocean Tuna
Comission.

\bibitem[\citeproctext]{ref-nootmornReproductiveBiologyYellowfin2005}
Nootmorn, Praulai, Anchalee Yakoh, and Kannokwan Kawises. 2005.
{``Reproductive Biology of Yellowfin Tuna in the Eastern {Indian
Ocean}.''} IOTC-2005-WPTT-14. Indian Ocean Tuna Comission.

\bibitem[\citeproctext]{ref-pecoraroPuttingAllPieces2017}
Pecoraro, C., I. Zudaire, N. Bodin, H. Murua, P. Taconet, P.
Díaz-Jaimes, A. Cariani, F. Tinti, and E. Chassot. 2017. {``Putting All
the Pieces Together: Integrating Current Knowledge of the Biology,
Ecology, Fisheries Status, Stock Structure and Management of Yellowfin
Tuna ({Thunnus} Albacares).''} \emph{Reviews in Fish Biology and
Fisheries} 27 (4): 811--41.
\url{https://doi.org/10.1007/s11160-016-9460-z}.

\bibitem[\citeproctext]{ref-puntSpatialStockAssessment2019}
Punt, André E. 2019. {``Spatial Stock Assessment Methods: {A} Viewpoint
on Current Issues and Assumptions.''} \emph{Fisheries Research} 213
(May): 132--43. \url{https://doi.org/10.1016/j.fishres.2019.01.014}.

\bibitem[\citeproctext]{ref-rogerRelationshipsYellowfinSkipjack1994}
Roger, Claude. 1994. {``Relationships Among Yellowfin and Skipjack Tuna,
Their Prey-Fish and Plankton in the Tropical Western {Indian Ocean}.''}
\emph{Fisheries Oceanography} 3 (2): 133--41.
\url{https://doi.org/10.1111/j.1365-2419.1994.tb00055.x}.

\bibitem[\citeproctext]{ref-sabarrosVerticalBehaviorHabitat2015}
Sabarros, Philippe, Evgeny Romanov, and Pascal Bach. 2015. {``Vertical
Behavior and Habitat Preferences of Yellowfin and Bigeye Tuna in the
{South West Indian Ocean} Inferred from {PSAT} Tagging Data.''}
IOTC--2015--WPTT17--42 Rev\_1. Indian Ocean Tuna Comission.

\bibitem[\citeproctext]{ref-urtizbereaProvidingScientificAdvice2020}
Urtizberea, Agurtzane, Massimiliano Cardinale, Henning Winker, Richard
D. Methot, Dan Fu, Toshihide Kitakado, Carmen Fernandez, and Gorka
Merino. 2020. {``Towards Providing Scientific Advice for {Indian Ocean}
Yellowfin in 2020.''} IOTC-2020-WPTT22(AS)-21. Indian Ocean Tuna
Comission.

\bibitem[\citeproctext]{ref-urtizbereaPreliminaryAssessmentIndian2019}
Urtizberea, Agurtzane, Dan Fu, Gorka Merino, Richard D Methot,
Massimiliano Cardinale, Henning Winker, John Walter, and Hilario Murua.
2019. {``Preliminary Assessment of {Indian Ocean} Yellowfin Tuna
1950-2018 ({Stock Synthesis}, V3.30).''} IOTC-2019-WPTT21-50. Indian
Ocean Tuna Comission.

\bibitem[\citeproctext]{ref-wainQuantifyingIncreaseFishing2021}
Wain, Gwenaëlle, Loreleï Guéry, David Michael Kaplan, and Daniel
Gaertner. 2021. {``Quantifying the Increase in Fishing Efficiency Due to
the Use of Drifting {FADs} Equipped with Echosounders in Tropical Tuna
Purse Seine Fisheries.''} Edited by Richard O'Driscoll. \emph{ICES
Journal of Marine Science} 78 (1): 235--45.
\url{https://doi.org/10.1093/icesjms/fsaa216}.

\bibitem[\citeproctext]{ref-zudairePreliminaryEstimatesSex2022}
Zudaire, Iker, Iraide Artetxe-Arrate, Jessica H. Farley, Hilario Murua,
Deniz Kukul, Annie Vidot, Shoaib Razzaque, et al. 2022. {``Preliminary
Estimates of Sex Ratio, Spawning Season, Batch Fecundity and Length at
Maturity for {Indian Ocean} Yellowfin Tuna.''} IOTC-2022-WPTT24(DP)-09.
Indian Ocean Tuna Comission.

\bibitem[\citeproctext]{ref-zudaireReproductivePotentialYellowfin2013}
Zudaire, Iker, Hilario Murua, Maitane Grande, and Nathalie Bodin. 2013.
{``Reproductive Potential of {Yellowfin Tuna} ({Thunnus} Albacares) in
the Western {Indian Ocean}.''} \emph{Fishery Bulletin} 111 (3): 252--64.
\url{https://doi.org/10.7755/FB.111.3.4}.

\end{CSLReferences}



\end{document}
